\section{Cultivation and differentiation of HAoSMCs}
\label{sec:cultivation}
For the following experiments human arortic smooth muscle cells were used. [Describe some properties of the cells]. Cells were kept at 37°C and 5\% CO2 when ever possibile.

Cells were differentiated using Cytokines -> reference to the introdcution and cited literature.

    \subsection{Thawing \& Cultivation}
    Cells were cultivated to a maximum passage of 10, after that new passage cells were thawed. For long time storage cells were kept in liquid nitrogen in [cryo medium]. When required new cells (6th passage) were need cells were thawed at 37°C in the water bath and transfered to a falcon. After centrifugation for 2 min at 300xg the supernatant was remoced and the cell pellet taken up in 14 mL of M231 + SMSG and cultivated in a T75 flask. Every other day 2/3 of the medium were removed and replaced by fresh.

    \subsection{Passaging}
    When reaching a maximum of ~80\% confluency (approx. once a week) the medium was removed completely and cells were washed once with 5 mL of PBS. Then the cells were incubation with 3 mL trypsin for 4 min at 37°C. After 7 mL M231 were added to the deattched cells and the cells were transfered to a falcon and pelleted for 4 min at 300xg. The supernatant was removed and the pellet resuspenden in M231 + SMGS, seeding ~$\num{500e3}$ cells per T75 flask.

    \subsection{Preparation of Collagen I matrix}
    Have a look at the protocol

    \subsection{Differentitation of cells}
    Differentiation was carried out in 24 wells plates with 1 mL M231 supplemented with 1 \% FBS and different cytokines:
    \begin{itemize}
        \item \textbf{Day 0:} Matrix and cells were prepareed as described in the sections Preparation of Col I matrix and Passaging. Seeding $\num{40e3}$ in M231 + SMGS after hardening of 160 µL collagen 1 matrix.
        \item \textbf{Day 1:} After ~24 h the medium was replaced with 1 mL M231 + 1\% FBS + ng/mL TGFb (or just 1 mL M231 + 1\% FBS).
        \item \textbf{Day 5:} The medium was replaced with 1 mL M231 + 1\% FBS + ng/mL IL-1 + ng/mL PDGF-BB (or just 1 mL M231 + 1\% FBS).
        \item \textbf{Day 7:} Potentially further stimulation described in the section of the used assay.
    \end{itemize}

\section{mRNA Quantification}
\label{sec:qpcr}
Some sentence ragarding the cells and primers and the method.

    \subsection{RNA Isolation}
    RNA was isolated using the kit and extraction was performed according to the corresponding protocol, using an extra washing step with Ethanol and eluting with 20 µL of RNase-free water. Determination of nucleic acid concentration was carried out with the NanoDrop.

    \subsection{Reverse Transcription}
    For reverse transcription RNA samples were diluted to yield 10 µL of ng/µL RNA. The samples were heated for 5 min at 68°C before adding 10 µL of the RT reaction mix described in the following table:

    The reaction was carried out for 60 min at 37°C before inactivating the enzyme for 5 min at 95°C. cDNA was used for qPCR or stored at -20°C.

    \subsection{qPCR}
    qPCR was performed for CNN1 and MMP9, using GAPDH mRNA levels as a reference. The analysis was performed with 4 biological replicated with 3 technical replicates each.

    \subsection{Processing of data}
    Data visualization and statistical analysis was done using python and the modules: pandas, numpy, scipy as well as pyplot and seaborn. Assuming a normal distribution, student's t-test was used, a p-value of 0.05 is considered as significant. For detailed information please check the script.

\section{Energy Profiling}
\label{sec:seahorse}
Short description of the method and again the cells that were used.

    \subsection{Seahorse Assay}
    - See the protocol written by Tobi and adapted by me. Also website?

    \subsection{Processing of data}
    - Describe and link the script

\section{Oxidative Stress Assay}
\label{sec:cellrox}
How does it work in general?
Used with different conditions. First just the boost, than titration, finally the quench.

    \subsection{CellROX Assay}
    - Wash
    - Add stuff
    - Wait
    - Image

    \subsection{Boost with PDGF-BB}
    Describe the different conentrations and evaliation of data.

    \subsection{Time Titration and PDGF-BB Titration}
    Describe the different conditions and evaluation of data.

    \subsection{Quench with NAC}
    Short descrription of NAC and description of conditions and evaluation of data

\section{Immunoflourescence}
\label{sec:if}
Fibronektion as marker of matrix. Used cells.
Maybe also the anti-8-oxoG AB?

    \subsection{Protocol}
    Fixate, do a lot of steps.

    \subsection{Processing of data}
    Me counting pixels.

\section{Curation of Data for postGWAS analyses}
\label{sec:database}
Describe how we get the data. REST APIs, FTP server.
Describe a relational database (sqlite3)

Describe all the data that we have, where we get it and how we process it before entering it into the database.

\section{Visualization of GWAS data}
\label{sec:gwas_vis}
Some general words regarding bokeh webserver.
Everything we have is split into the main file that takes care of all the visualization and the data provider that serves as a kind of backend and provides then main file with the required data.

\section{Enrichment analysis}
\label{sec:enrichment}
- Theory, what is happening
- The Analysis
- Visualization
