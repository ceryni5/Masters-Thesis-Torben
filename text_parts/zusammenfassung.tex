Nach Angaben der Weltgesundheitsorganisation (WHO), sind „Herz-Kreislauf-Erkrankungen (KHK) [...] die weltweit häufigste Todesursache und fordern jedes Jahr schätzungsweise 17,9 Millionen Todesopfer.“ \cite{whoCardiovascularDiseases2022}. Ein entscheidender Schritt für die Entwicklung erfolgreicher Therapieansätze ist die Aufklärung der zugrunde liegenden molekularen Prozesse.\\
Glatten Gefäßmuskelzellen (gGMZ) spielen eine wichtige \cite{doranRoleSmoothMuscle2008}, aber nur unvollständig verstandene Rolle in der Entstehung und dem Erhalt von atherosklerotischen Plaques \cite{liuSmoothMuscleCell2019, grootaertVascularSmoothMuscle2021, yapSixShadesVascular2021}. Wir stellen die Hypothese auf, dass die anhaltende Stimulation von gGMZ mit proinflammatorischen Zytokinen, wie z. B. PDGF-BB, oxidativen Stress verursacht und so zur Pathogenese beiträgt.\\
Wir kommen zu dem Schluss, dass das von uns genutzte CellROX\texttrademark~Assay eine geeignete Methode zur Evaluation dieser Hypothese ist und bereiten Vorschläge für weitere Optimierungen des Assays. Zusätzlich zeigen wir, dass die Inkubation von \textit{in vitro} dedifferenzierten HAoSMCs mit hohen Konzentrationen an PDGF-BB die Produktion von reaktiven Sauerstoffspezies (ROS) induziert.\\
Ergänzt zu diesem Hypothesen-getriebenen Ansatz konsultieren wir Daten aus Hochdurchsatz-Verfahren in Form von Genomweite Assoziationsstudien (GWAS). GWAS sind ein Kernstück der modernen Genetik und eröffnen grandiose Möglichkeiten zur Identifikation krankheitsassoziierter Loci \cite{uffelmannGenomewideAssociationStudies2021}. Für ein besseres Verständnis des genomischen Kontexts von KHK Risikovarianten präsentieren wir den GWAS Navigator - eine web basierte, interaktive Anwendung, die GWAS Daten des CARDIoGRAMplusC4D Konsortiums \cite{aragamDiscoverySystematicCharacterization2021} in Verbindung mit kurierten regulatorischen Elementen visualisiert. \\
Abschließend nutzen wir die gesammelten Daten für eine Anreicherungsanalyse, in der wir für die Überschneidung von KHK Risikovarianten mit gewebe- oder zellspezifischen regulatorischen Elementen testen. Wir beobachten eine signifikante Überrepräsentation solcher Überschneidungen vor allem in Proben des Herzens, der Lunge und von arteriellem Gewebe. Ein Ergebnis, das die oft beschriebene Assoziation von Herz- mit Lungenerkrankungen unterstreicht \cite{carterAssociationCardiovascularDisease2019, nowakLungFunctionCoronary2018, hanPulmonaryDiseasesHeart2007}.
