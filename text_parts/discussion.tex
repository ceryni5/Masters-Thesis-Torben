\section{PDGF-BB Signaling Seems to Induce a Synthetic Phenotype in HAoSMCs}
The crucial role of \acp{vsmc} in atherogenesis has been the subject of intense research for the last few decades \cite{grootaertVascularSmoothMuscle2021, yapSixShadesVascular2021}. While it has traditionally been assumed that \acp{vsmc} adopt a protective role by stabilizing the arteriogenic plaque. This model is rapidly evolving and starting to consider the existence of a diverse set of dedifferentiated phenotypes \cite{liuSmoothMuscleCell2019}. A central hub of this process is an initial dedifferentiated mesenchymal-like phenotype, that displays a proliferative phenotype and reduced expression of contractile markers \cite{yapSixShadesVascular2021}. It is thought to be initiated by the \ac{tf} \ac{klf4}, which induces expression of mesenchymal markers such as \ac{sca1} \cite{yapSixShadesVascular2021}. Amongst other pathways, expression of \ac{klf4} can be induced by \ac{pdgf} signaling \cite{liuKruppellikeFactorAbrogates2005} via \ac{sp1} \cite{deatonSp1dependentActivationKLF42009}. Additionally, \ac{pdgf} suppresses the contractile phenotype by phosphorylation of \ac{elk-1} \cite{wangMyocardinTernaryComplex2004} as well as the expression of \ac{dock2} \cite{guoDedicatorCytokinesisNovel2015}, both of which disrupt myocardin/\ac{srf} mediated expression of contractile genes. The mesenchymal-like phenotype is postulated to serve as a precursor for other dedifferentiated \acp{vsmc} phenotypes \cite{yapSixShadesVascular2021}.\\
The contractile expression profile of differentiated \acp{vsmc} is constantly maintained by myocardin/\ac{srf} signaling \cite{longMyocardinSufficientSmooth2008} and external stimulation by the \ac{ecm} and cytokines such as \ac{tgf} \cite{davis-dusenberyDownregulationKruppellikeFactor42011}. \acp{haosmc} used in this thesis, seem to have initially adopted a dedifferentiated phenotype, characterized by the loss of contractile marker \ac{cnn1} \cite{owensMolecularRegulationVascular2004} (figure \ref{fig:qPCR} top). When stimulated for two days with \ac{tgf}, \acp{haosmc} display increased expression of \ac{cnn1}. Additionally, this phenotype shows a significant decrease in  basal mitochondrial respiration, ATP production and maximal respiration (figure \ref{fig:energy_profile} B top), possibly adapting to the energetic needs of the contractile phenotype, which is considered to be quiescent \cite{dobnikarDiseaserelevantTranscriptionalSignatures2018}. Further simulation for 4 additional days with \ac{pdgf} \& \ac{il1}, yields a (not quite significant after four biological repeats, $p=0.087$) drop in \ac{cnn1} expression (figure \ref{fig:qPCR}) as well as a rebound of basal mitochondrial respiration, \ac{atp} production as well as maximal respiration to similar levels as for initially dedifferentiated VSMCs (figure \ref{fig:energy_profile} B bottom).\\
Another important aspect of phenotypic transition and plaque development is the remodeling of the \ac{ecm} by \acp{mmp} \cite{johnsonMetalloproteinasesAtherosclerosis2017}. While not significant in 4 biological repeats, \ac{pdgf}-induced dedifferentiation seems to increase the expression of \ac{mmp9} for \acp{haosmc} cultivated on \ac{col1} matrix (figure \ref{fig:qPCR} bottom). \ac{mmp9} is an important component of atherosclerogenesis \cite{galisIncreasedExpressionMatrix1994} and a biomarker for advanced atherosclerotic lesions \cite{langleyExtracellularMatrixProteomics2017}. The fact that this trend is only observable for cells cultivated on \ac{col1} (figure \ref{fig:qPCR} bottom, p = 0.063), underlines the bi-directionality of the \ac{ecm}-\ac{vsmc}-interactions and the complexity of \ac{vsmc} dedifferentiation.\\
Of course the ac{pdgf}-induced phenotype can not be grasped with only two markers and requires a more indepth analysis.


\section{CellROX\texttrademark~Green is Suitable for Assessing ROS Generation in HAoSMCs}
Evaluating the response to further stimulation with \ac{pdgf}, the CellROX\texttrademark~Assay was able to confirm a result previously observed in the group (unpublished). Stimulation of \acp{haosmc} cultivated on \ac{col1} matrix and stimulated for 2 days with \ac{tgf} and 4 days with \ac{pdgf} \& \ac{il1}, are susceptible to the generation of \ac{ros} by \ac{pdgf} boost (figure \ref{fig:cellrox_8con}). Further evaluating the limits of the used assay, it is obvious, that a threshold concentration of 200\,ng/ml \ac{pdgf} is required to induce a significant increase in signal over the negative control (0\,ng/mL) (figure \ref{fig:cellROX_titration}). It was further observed, that the signal highly depends on the incubation time. While the trend for each biological repeat is clear, the variance between repeats is almost as high. The assay is working but could greatly benefit from retroactive normalization (figure \ref{fig:cellROX_titration_norm}) of further optimization towards reproducibility - reducing the required amount of required biological repeats. A potential parameter to explore is the use of different CellROX\texttrademark~Green concentrations. Finally, a recovery experiment was performed. Before and during the boost, cells were co-incubated with \ac{nac}, a potent antioxidant. While not significant after 3 biological repeats, a strong trend was observable, that cells treated with \ac{nac} show no CellROX\texttrademark~Green signal, supporting the expectation that the observed signal is indeed due to the generation of \ac{ros} (figure \ref{fig:cellROX_titration_norm}).\\
Moreover, it needs to be evaluated if the used \ac{pdgf} concentration of 200\,ng/ml ($\widehat{=}$8.25\,nM) is physiologically relevant. Unfortunately, cytokine concentrations are usually assessed as plasma concentrations and no \textit{in vivo} data for local concentrations during paracrine signaling exists. While the manufacturer describes the \ac{ec50} for \ac{pdgf}-induced proliferation of Balb/c 3T3 cells between 1.0\,-\,3.0\,ng/mL \cite{peprotecheclimitedRecombinantHumanPDGFBB2022}, higher concentrations have frequently been used in the literature. \textcite{gravesPlateletderivedGrowthFactor1996a} observed increased formation of \ac{camp} until 10 nM ($\widehat{=}$240\,ng/mL) \ac{pdgf} when assessing the dose-response relationship between \ac{camp} formation after \ac{pdgf} stimulation of SMCs. \textcite{newmanMultipleCellTypes2021a} use 50,\ng/mL \ac{pdgf} for the differentiation of murine \acp{vsmc} in the context of atherosclerosis, and \textcite{bouziguesRegulationROSResponse2014a} identified 100\,ng/mL as a saturating concentration for the generation of \ac{h2o2} as a response to \ac{pdgf} signaling in \acp{vsmc}.\\
The next up-and-coming experiment would be the rescue experiment to confirm that the generation of \ac{ros} is indeed caused by \ac{pdgf} stimulation. Namely by the knockdown of the \ac{pdgfr}\beta. The same approach could be pursued to study downstream factors of \ac{pdgfr} signaling that are involved in the generation of \ac{ros}. An exemplary candidate would be \ac{stat1}, a \ac{tf} whichs deletion reduces plaque formation during atherogenesis and is a required component of \ac{pdgf_simple}-signaling induced inflammation \cite{hePDGFRvSignallingRegulates2015}. In addition to its genomic function, it can be imported into mitochondria where it interacts with respiratory complexes and triggers the generation of \ac{ros} \cite{wangSTATROSCycleExtends2018} during hepatic \cite{leeRoleSTAT1IRF12007} and \ac{ifn} induced cancer cell apoptosis \cite{wangSTATROSCycleExtends2018}.\\
Finally, it has to be addressed, that during the recovery experiment with \ac{nac}, the CellROX\texttrademark~Green signal would multiple times only develop only after cells were taken out of the controlled environment of 37°C and 5\,\%\ac{co2}. This suggests, that the \ac{pdgf} is not the sole trigger of \ac{ros} generation. To follow up on this idea, it would be beneficial to repeat the experiment under better-controlled conditions. We additionally tried to assess oxidative stress with an anti-8-oxoguanine antibody that detects 8-oxoguanine, a base modification caused by \ac{ros}. An attempt that unfortunately failed because the cultivation of \acp{haosmc} for 7 days in M231 + 1\,\%\,\ac{fbs} was sufficient to induce oxidative damage to the genome (results not shown -> maybe I'll still include them).


\section{The GWAS Navigator}
Like all primates, humans are extremely visual creatures. We have evolved specialized brain structures for the processing of visual stimuli \cite{kaasCurrentResearchOrganization2014}, granting us superior recognition of visual patterns \cite{mattsonSuperiorPatternProcessing2014}. Thus, making the tools for visualization of data, powerful and important resources for interactive exploration as well as scientific communication.\\
The \ac{gwas} Navigator was developed to display \ac{cad} \ac{gwas} summary statistics in an easily accessible format for medical researchers. In an iterative process, a multitude of possible implementation approaches was explored, finally arriving at the prototype presented in this thesis. At this point, the tool is built as a bokeh application (section \ref{sec:gwas_vis}) that dynamically fetches data from an SQLite database and renders it to the browser.\\
Databases are a structured collection of data and a stable of data science, providing many advantages over the storage of data in the form of spreadsheets such as access speed, maintainability, and multiuser access. They are designed to hold large collections of data and provide secure and fast access by querying via specifically designed database engines. Relational databases, like SQLite, are the most popular way of flexible representing data in the form of tables with columns and rows. They are usually queried and manipulated with commands using \ac{sql}, an internally consistent, human-readable programming language. \cite{oraclecorporationWhatDatabase2022, oraclecorporationWhatRelationalDatabase2022} SQLite is a public domain database engine that generates cross-platform, single file databases and is the most used database engine worldwide \cite{thesqliteconsortiumSQLite2022}.\\
While certainly not the only option, bokeh fulfills all the basic requirements for the task at hand. Combining the elegant visualization resources of rendering data with \ac{html}, \ac{css} \& \ac{js} to the browser with the powerful data processing capabilities of python. All bundled into one easy-to-learn ecosystem, providing a level of abstraction that is required for the construction of a prototype. Additionally, the bokeh server makes the application easily deployable for potential use on a local network \cite{bokehdevelopmentteamBokehPythonLibrary2022}. \\
Overall the \ac{gwas} Navigator grants a first glance at the genomic context of disease-associated genomic loci. The next step in its development should undoubtedly be the local deployment for the rest of the lab. It provides basic functionality and the possibility for implementation of many additional features. Reaching from basic improvements to usability in the formed tissue-specific annotations to the displayed tracks and the selection tool, to the expansion with new datasets.


\section{Overlap of CAD Associated Variants with Regulatory Elements is Enriched in Heart, Artery \& Lung Tissue}
The database makes all the collected not only easily accessible for visualization purposes, but also follow-up studies. The curated data is utilized in an initial postGWAS analysis, scanning for biosamples with \acp{cCRE} enriched in \ac{cad} \ac{gwas} variants via Fisher's exact test. This way identifying 34 biosamples (of [NUMBER] tested) that show significant overrepresentation (figure \ref{fig:enrichment_scatter} \& table \ref{tab:enriched_tissues}). After annotation of these biosamples, over 40\,\% (14/34) of enriched biosamples stem from heart or artery tissue and are therefore directly affected by arteriosclerosis. An additional 20\,\% stem from lung tissue, an observation in line with the often reported association between heart- and lung disease \cite{carterAssociationCardiovascularDisease2019, hanPulmonaryDiseasesHeart2007}. The association of heart- and lung disease prevails even after adjustment for shared risk factors such as tobacco or age. Additionally, \textcite{auyeungAssociationGeneticInstrumental2018} were recently able to demonstrate that greater \ac{fev1} decreases the risk of \ac{cad} via Mendelian randomization. Still, the causality of this relationship remains unclear. While it is tempting to speculate that impaired lung function or systematic inflammation by chronic diseases like \ac{copd} result in an elevated risk for cardiovascular diseases, such hypotheses are difficult to evaluate due to reverse causation \cite{nowakLungFunctionCoronary2018}. \ac{cad} might also be the risk factor for lung diseases or both pathologies could share additional not properly adjusted confounding factors. Similarly, the identification of lung tissue in our analysis might hint at the involvement of the lung during the development of \ac{cad} or a shared genomic predisposition of heart- and lung disease. Following up on the topic of systemic inflammation, the immune cells in which \acp{cCRE} enrich are CD14+ monocytes (table \ref{}), a cell type that is known for the secretion of proinflammatory cytokines during injury or inflammation \cite{kapellosHumanMonocyteSubsets2019}. Interestingly, CD14++CD16+CCR2+ \& CD14++CD16-CCR2+ monocytes show significantly higher counts in patients with acute \ac{hf} over patients with stable \ac{hf} or \ac{cad} \cite{wrigleyCD14CD16Monocytes2013}.\\
Finally, the same method and already collected data could be applied to check for the overlap of disease-associated variants with the enhancers identified as part of the \ac{abc} model. Further, using the enhancer-promotor connections, to identify potentially affected genes.
