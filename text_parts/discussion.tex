\section{PDGF-BB signaling induces a synthetic phenotype in HAoSMCs}
The crucial role of \acp{vsmc} in atherogenesis has been the subject of extensive research for the last few decades \cite{grootaertVascularSmoothMuscle2021, yapSixShadesVascular2021}. Traditionally it has been assumed that \acp{vsmc} adopt a protective role by stabilizing the atherosclerotic plaque. This model is rapidly evolving and starting to consider the existence of a diverse set of dedifferentiated phenotypes \cite{liuSmoothMuscleCell2019}. A central hub of the dedifferentiation process is an initial mesenchymal-like phenotype. This phenotype exhibits a proliferative phenotype and reduced expression of contractile markers \cite{yapSixShadesVascular2021}. This phenotype is thought to be initiated by the \ac{tf} \ac{klf4}, which induces the expression of mesenchymal markers such as \ac{sca1} \cite{yapSixShadesVascular2021}. Amongst other pathways, expression of \ac{klf4} can be induced by \ac{pdgf} signaling \cite{liuKruppellikeFactorAbrogates2005} via \ac{sp1} \cite{deatonSp1dependentActivationKLF42009}. Additionally, \ac{pdgf} suppresses the contractile phenotype by phosphorylation of \ac{elk-1} \cite{wangMyocardinTernaryComplex2004} as well as the expression of \ac{dock2} \cite{guoDedicatorCytokinesisNovel2015}. Both processes disrupt myocardin/\ac{srf} mediated expression of contractile genes. The mesenchymal-like phenotype is postulated as a precursor for other dedifferentiated \acp{vsmc} phenotypes \cite{yapSixShadesVascular2021}.\\
The contractile differentiated \acp{vsmc} phenotype is constantly maintained by myocardin/\ac{srf} signaling \cite{longMyocardinSufficientSmooth2008}, as well as external stimulation by the \ac{ecm} and cytokines such as \ac{tgf} \cite{davis-dusenberyDownregulationKruppellikeFactor42011}. \acp{haosmc} used in this thesis seem to have initially adopted a dedifferentiated phenotype, characterized by the loss of contractile marker \ac{cnn1} \cite{owensMolecularRegulationVascular2004} (figure \ref{fig:qPCR_result} top). When stimulated with \ac{tgf} for four days, \acp{haosmc} display increased expression of \ac{cnn1}. Additionally, this phenotype shows a significant decrease in  basal mitochondrial respiration, ATP production, and maximal respiration (figure \ref{fig:energy_profile} B). This trend is possibly and adaptation to the energetic needs of the contractile phenotype, which is considered quiescent \cite{dobnikarDiseaserelevantTranscriptionalSignatures2018}. Further simulation with \ac{pdgf} and \ac{il1} for two days yields a the decrease expression of \ac{cnn1}, on the verge of significance (p=0.087). Moreover, the energy metabolism changes again, characterized by a rebound of basal mitochondrial respiration, \ac{atp} production, and maximal respiration to similar levels as initially dedifferentiated VSMCs (figure \ref{fig:energy_profile} B).\\
Another important aspect of phenotypic transition and plaque development is the remodeling of the \ac{ecm} by \acp{mmp} \cite{johnsonMetalloproteinasesAtherosclerosis2017}.
Our experiments hint toward a possible increase of \ac{mmp9} expression for \acp{haosmc} cultivated on \ac{col1} matrix (figure \ref{fig:qPCR_result} bottom) after \ac{pdgf}-induced dedifferentiation. However, this trend remained not significant in four biological replicates(p = 0.063). \ac{mmp9} is an important component of atherosclerogenesis \cite{galisIncreasedExpressionMatrix1994} and a biomarker for advanced atherosclerotic lesions \cite{langleyExtracellularMatrixProteomics2017}. The fact that this trend is only observable for cells cultivated on \ac{col1} (figure \ref{fig:qPCR_result} bottom) underlines the bi-directionality of the \ac{ecm}-\ac{vsmc}-interactions and the complexity of \ac{vsmc} dedifferentiation.\\
Of course, the \ac{pdgf}-induced phenotype can not be grasped with only two markers and requires a more in-depth analysis, for example via \ac{rna}-sequencing.


\section{CellROX\texttrademark~Green is suitable for assessing ROS generation in HAoSMCs}
Evaluating the response to further stimulation with \ac{pdgf}, the CellROX\texttrademark~Assay confirmed a result previously observed in the group (unpublished). Stimulation of \acp{haosmc} cultivated on \ac{col1} matrix and treated for four days with \ac{tgf} and two days with \ac{pdgf} and \ac{il1} are susceptible to the generation of \ac{ros} by \ac{pdgf} boost (figure \ref{fig:cellrox_8con}).\\
Further evaluating the limits of the used assay, it is obvious that a threshold concentration of 200\,ng/ml \ac{pdgf} is required to induce a significant increase in signal over the negative control (0\,ng/mL) (figure \ref{fig:cellROX_titration}). In addition, it was observed that the signal intensity highly depends on the incubation time. While the trend for each biological repeat is clear, the variance between repeats is in a similar range. The assay is working reliably but could greatly benefit from retroactive normalization (figure \ref{fig:cellROX_titration_norm}) or further optimization of reproducibility - reducing the required amount of biological repeats. An alternative option for normalization could be provided by direct stimulation with \ac{h2o2} to be used as a reference. Finally, the robustness of the assay may be improved by the exploration of different CellROX\texttrademark~Green concentrations.\\
Finally, a recovery experiment was performed. Before and during the boost, cells were co-incubated with \ac{nac}. \ac{nac} is a popular and potent antioxidant used in cell culture experiments, and it likely acts by being metabolized into sulfane sulfur species that scavenge \ac{ros} in the mitochondria \cite{ezerinaNAcetylCysteineFunctions2018}. Cells treated with \ac{nac} showed only little to no CellROX\texttrademark~Green signal. Even though this trend remained not significant after three replicates, it supports the expectation that the observed signal is indeed due to the generation of \ac{ros} (figure \ref{fig:cellROX_titration_norm}).\\
Moreover, it needs to be evaluated if the used \ac{pdgf} concentration of 200\,ng/ml ($\widehat{=}$8.25\,nM) is physiologically relevant. Unfortunately, cytokine concentrations are usually assessed as plasma concentrations, and no \textit{in vivo} data exists for local concentrations during paracrine signaling. While the manufacturer describes the \ac{ec50} for \ac{pdgf}-induced proliferation of Balb/c 3T3 cells between 1.0\,-\,3.0\,ng/mL \cite{peprotecheclimitedRecombinantHumanPDGFBB2022}, higher concentrations have frequently been used in the literature. \textcite{gravesPlateletderivedGrowthFactor1996a} observed increased formation of \ac{camp} up to 10 nM ($\widehat{=}$240\,ng/mL) \ac{pdgf} when assessing the dose-response relationship between \ac{camp} formation after \ac{pdgf} stimulation of SMCs. \textcite{newmanMultipleCellTypes2021a} used 50\,ng/mL \ac{pdgf} for the differentiation of murine \acp{vsmc} in the context of atherosclerosis, and \textcite{bouziguesRegulationROSResponse2014a} identified 100\,ng/mL as a saturating concentration for the generation of \ac{h2o2} as a response to \ac{pdgf} signaling in \acp{vsmc}.\\
The next up-and-coming experiment would be the rescue experiment to confirm that the generation of \ac{ros} is indeed caused by \ac{pdgf} stimulation. Namely by the knockdown of the \ac{pdgfr}\beta. The same approach could be pursued to study downstream factors of \ac{pdgfr} signaling that are involved in the generation of \ac{ros}. An exemplary candidate would be the transcription factor \ac{stat1}, which upon deletion, reduces plaque formation during atherogenesis and is a required component of \ac{pdgf_simple}-signaling induced inflammation \cite{hePDGFRbetaSignallingRegulates2015}. In addition to its genomic function, \ac{stat1} can be imported into mitochondria, where it interacts with respiratory complexes and triggers the generation of \ac{ros} \cite{wangSTATROSCycleExtends2018} during hepatic apoptosis \cite{leeRoleSTAT1IRF12007} and \ac{ifn} induced cancer cell apoptosis \cite{wangSTATROSCycleExtends2018}.\\
Finally, it has to be addressed that during the recovery experiment with \ac{nac}, the CellROX\texttrademark~Green signal would occasionally only develops outside the controlled environment in the incubator. This suggests that the \ac{pdgf} is not the sole trigger of \ac{ros} generation. Repeating the experiment under better-controlled conditions would be a great option to follow up on this idea\\
We additionally tried to assess oxidative stress with an anti-8-oxoguanine antibody that detects 8-oxoguanine, a base modification often observed in the presence of \ac{ros} \cite{leon8OxoguanineAccumulationMitochondrial2016}. This attempt was not successful because the starvation of \acp{haosmc} for seven days in M231 supplemented with 1\,\%\,\ac{fbs} was sufficient to induce oxidative damage to the genome (figure \ref{fig:antibody}).


\section{The GWAS Navigator}
Like all primates, humans are extremely visual creatures. We have evolved specialized brain structures for processing visual stimuli \cite{kaasCurrentResearchOrganization2014}, granting us superior recognition of visual patterns \cite{mattsonSuperiorPatternProcessing2014}. Thus, making data visualization tools powerful and important resources for interactive data exploration and scientific communication.\\
The \aca{gwas} Navigator was developed to display CARDIoGRAMplusC4D Consortium \cite{aragamDiscoverySystematicCharacterization2021s} summary statistics in a comprehensive and visually appealing format for medical researchers. In an iterative process, many possible implementation approaches were explored, finally resulting in the prototype presented in this thesis. At this point, the tool is built as a bokeh application (section \ref{sec:gwas_vis}) that dynamically fetches data from an SQLite database and renders it to the browser.\\
Databases are a structured data collection at the heart of data science. Databases provide many advantages over data stored in spreadsheets, such as access speed, maintainability, and multiuser access. They are designed to hold large data collections and provide secure and fast access by querying via specifically designed database engines. Relational databases, like SQLite, are the most popular way for flexibly representing data in tables comprising columns and rows. They are usually queried and manipulated with commands in \ac{sql}, an internally consistent, human-readable programming language. \cite{oraclecorporationWhatDatabase2022, oraclecorporationWhatRelationalDatabase2022} SQLite is a public domain database engine that generates cross-platform, single file databases and is the most used database engine worldwide \cite{thesqliteconsortiumSQLite2022}.\\
While certainly not the only option, bokeh fulfills all the basic requirements for the task. The package combines the elegant visualization resources for rendering data with \ac{html}, \ac{css}, and \ac{js} to the browser with the powerful data processing capabilities of python. All are bundled into one easy-to-learn ecosystem, providing a level of abstraction required for a prototype's construction. Additionally, the bokeh server makes the application easily deployable for potential use on a local network \cite{bokehdevelopmentteamBokehPythonLibrary2022}. \\
To summarize, the \aca{gwas} Navigator grants an overview of the genomic context of disease-associated genomic loci. The next step in its development should undoubtedly be the local deployment for the rest of the group. It provides basic functionality and the possibility to implement many additional features. Options range from basic improvements to usability in the formed tissue-specific annotations to the displayed tracks and the selection tool to the adoption of new datasets.


\section{Overlap of CAD associated variants with regulatory elements is enriched in heart, artery, and lung tissue}
In addition to providing the basis for the \aca{gwas} Navigator, the database also makes the data easily accessible for follow-up studies. The curated data are utilized in an initial postGWAS analysis, scanning for biosamples with \acp{cCRE} enriched in CARDIoGRAMplusC4D Consortium \cite{aragamDiscoverySystematicCharacterization2021s} proxy variants via Fisher's exact test. This way identifying 34 biosamples (of 1257 tested) that show significant overrepresentation (figure \ref{fig:enrichment_scatter} and table \ref{tab:enriched_tissues}). After annotating these biosamples, over 40\,\% (14/34) of enriched biosamples stem from heart or artery tissue and are directly affected by atherosclerosis. An additional 20\,\% was annotated to stem from lung tissue, an observation in line with the frequently reported association between heart- and lung diseases \cite{carterAssociationCardiovascularDisease2019, hanPulmonaryDiseasesHeart2007}. The association of heart- and lung diseases prevails even after adjustment for shared risk factors such as tobacco usage or age. Additionally, \textcite{auyeungAssociationGeneticInstrumental2018} demonstrated via Mendelian randomization that greater \ac{fev1} decreases the risk of \ac{cad}. Still, the causality of this relationship remains unclear. While it is tempting to speculate that impaired lung function or systematic inflammation by \ac{copd} results in an elevated risk for cardiovascular diseases, such hypotheses are difficult to evaluate due to reverse causation \cite{nowakLungFunctionCoronary2018}. \ac{cad} might also be a risk factor for lung diseases, or both pathologies could share additional mutually relevant confounding factors. Similarly, the identification of lung tissue in our analysis might point toward the contribution of the lung during the development of \ac{cad} or a shared genomic predisposition of heart- and lung disease. Following up on systemic inflammation, the immune cells in which \acp{cCRE} enrich are CD14\textsuperscript{+} monocytes (table \ref{tab:enriched_tissues_all}), a cell type known for the secretion of proinflammatory cytokines during injury or inflammation \cite{kapellosHumanMonocyteSubsets2019}. Interestingly, CD14\textsuperscript{++}CD16\textsuperscript{+}CCR2\textsuperscript{+} and CD14\textsuperscript{++}CD16\textsuperscript{-}CCR2\textsuperscript{+} monocytes show significantly higher counts in patients with acute \ac{hf} over patients with stable \ac{hf} or \ac{cad} \cite{wrigleyCD14CD16Monocytes2013}.
\pagebreak
Finally, the same method and already collected data could be applied to check for the overlap of disease-associated variants with the enhancers identified as part of the \ac{abc} model. Consequently, based on the enhancer promotor connection, one may be able to identify potentially affected genes.
