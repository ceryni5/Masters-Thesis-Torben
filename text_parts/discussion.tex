\section{PDGF-BB Signaling Seems to Induce a Synthetic Phenotype in HAoSMCs}

The crucial role of \acp{vsmc} in atherogenesis has been the subject of intense research for the last few decades \cite{}. While its has traditionally been assumed that they adopt a protective role by stabilizing the artheriogenic plaque. This model is rapidly evolving and starting to consider the existence of a diverse set of dedifferentiated phenotypes \cite{}. A central hub of this process is a mesenchymal-like phenotype, that . It is thought to be positively regulated by the \ac{tf} \ac{klf4}, which [], . Amongst other pathways, \ac{klf4} signaling is an important part in PDGF signaling \cite{}. Additionally \ac{pdgf} suppresses the expression of contractile genes via phosphorylation of ELK1, that then competes with MYOCD, a potent \cite{}. The mesenchymal-like phenotype is postulated to serve as a precursor for other phenotypes \cite{yapSixShadesVascular2021}.

To maintain their characteristic contractile phenotype, \acp{vsmc} require constant stimulation in the form of [] \cite{}. \acp{haosmc} used in this thesis, seem to have initially adopted a dedifferentiated phenotype, characterized by the loss of contractile marker \ac{cnn1} \cite{CNN1 as contractile marker in yapSixShadesVascular2021} (figure \ref{} top). When stimulated for two days with \ac{tgf}, a potent external stimulus for maintaining or inducing a contractile phenotype \cite{}, \acp{haosmc} indeed display increased expression of \ac{cnn1}. Additionally this phenotype displays a signigicant decrease in  basal mitochondrial respiration, ATP production as well as maximal respiration (figure \ref{} top), possibly adapting to the energetic needs of the contractile phenotype, that is considered to be considered to be quiescent \cite{get the source from yapSixShadesVascular2021}. Further simulation for 4 additonal days with \ac{pdgf} \& \ac{il1}, yields a (not quite significant after four biological repeats, $p=0.067$) drop in \ac{cnn1} expression (figure \ref{}) as well as an rebound of basal mitochondrial respiration, \ac{atp} production as well as maximal respiration to similar levels as for initially dedifferentiated VSMCs (figure \ref{} bottom). An additional property of the mesenchyml-like phenotype controlled by \ac{klf4} signaling is the uregulation of factors. Indeed . The changes in \ac{mmp9} expression seem to depend on the cultivation of \acp{haosmc} on the \ac{col1} matrix, underlying the fact, that the dedifferentiation of \acp{vsmc} is an highly complex process, mediated by a multitude of factors.

\section{CellROX\texttrademark~Green is Suitable for Assessing ROS Generation in HAoSMCs}
Evaluating the response to further stimulation with \ac{pdgf}, the CellROX\texttrademark~Assay was able to confirm a result previously pberserved in the group (unpublished). Stimulation of \acp{haosmc} cultivated un \ac{col1} matrix and stimulated for 2 days with \ac{tgf} and 4 days with \ac{pdgf} \& \ac{il1}, induce the generation of \ac{ros}. Further evaluating the limits of the used assay, it is obvious, that that a threshhold concentration of 200\,ng/ml \ac{pdgf} is required to induce significant increase in signal over the negative control (0,ng/mL). It was further observed, that the signal highly depends on the incubation time. While the trend for each biological repeat is clear, the variance between repeats varies almost as much. The assay is working but could greatly benefit from retroactively normalization of further optimization towards reproducablilty, this way reducing the amount of required biological. A potential parameter that could be explored is the use of different CellROX\texttrademark~Green concentrations. In addition, a recovery experiemnt was performed. Before and during the boost, cells were coincubated with \ac{nac}, a potent antioxidnant. While not significant after 3 biological repeats, an obvious trend was obserable, that cells treated with \ac{nac} show no signal, supporting the expectation that observed signal is indeed due to generation of \ac{ros}.\\
The next up and coming experiment would be the
- recovery experiment for PDGF-BB
- similar assay could be used to further evaluate the study of ROS generation due to \ac{pdgf} stimulation, screening different down-stream effectors of \ac{pdgfr} signaling. For example STAT1 (prominent overlap) has been described to play as significant role in \ac{pdgf} indcued pro-inflammatory signaling \cite{find source in hePDGFRvSignallingRegulates2015}. As well as ROS generation as a secondary messanger \cite{}.
This line of thought directly opens up the question of biological relevance. Is the used concentration of 200\,ng/ml ($\widehat{=}$8.25\,nM) \ac{pdgf} physiological? Unfortnataly, cytokine concentrations are usually assessed as plasma concentrations and no \textit{in vivo} data for local concentrations during paracrine signaling exists. While the manufacturer describes the \ac{ec50} for proliferation of Balb/c 3T3 cells between 1.0\,-\,3.0\,ng/mL \cite{}. The saturation concentration for \ac{cmap} generation in [cells] has been observed to be around nM ($\widehat{=}$8.25\,ng/mL) \cite{}. Further concentrations in the same order of magintude have also been used by \textcite{} for the study of \acp{vsmc} in the context of artheriosclerosis. \textcite{} describe 100\,ng/mL as the saturting conentration for the evaluation of \ac{h2o2} generation in \acp{vsmc} as reponse to \ac{pdgf} signaling.

Finally it has to be addressed, that for the recovery experiment, it was observed multiple times, that CellROX\texttrademark~Green signal would develop only after cells were taken out of the controlled environment of 37°C and 5\,\%\ac{co2}. This suggest, that the \ac{pdgf} is not the sole trigger of \ac{ros} generation. To follow-up on this idea, it would be beneficial to repeat the experiment under controlled conditions.


\section{The GWAS Navigator}
As all primates, human extremly visual creatures. Humans have evolved specialized brain strcutres for the processing of visual stimuli \cite{https://www.ncbi.nlm.nih.gov/pmc/articles/PMC4574956/}. Allowing us to superior recognition of visual patterns \cite{https://www.ncbi.nlm.nih.gov/pmc/articles/PMC4141622/}. Thus, making the tools for visualization of data, powerful and important resources for interactive exploration as well as science communication.\\
The \ac{gwas} Navigator was developed in to display \ac{cad} \ac{gwas} summary statistics in an easily accessable format for mediacl researchers. In an iterative process, a multitude of possible implementation approaches were explored, finally arriving at the prototype presented in this thesis. At this point in time the tool is build as an bokeh application (see section \ref{}) that dynamically fetches data from an \ac{sql} data base and renders it to the browser.

Databases are a stable of datascience. Relational databases as \ac{sql} databases, provide many adcantages over storing data as plain text. Namely

While certainly not the only option, bokeh fullfills all the basic requirements of the task at hand. Combining the powerful . All combined into one easy to learn ecosystems, providing a level of abstraction that is necassary that is required for the construction of a prototype, while maintaining low level access required for certain custumizations.

At this moment the \ac{gwas} Navigator fullfills all the basic needs to get a quick glance in the genomic conext  especially regulatory features in proxmity. The next step in its development is certainly the deployment to an , building on the established bokeh ecosystem. [Describe how to host an server] Further the tool is still a prototype,
-> other features: improve usablitilty by annotation of data
-> completely new data, maybe in the form of eQTLs

The GWAS Navigator was developed as a tool to make CAD GWAS summary statistcis easily and readily accessable for medical researchers. In an iterative process, multiple approaches were implemented, finally arriving at the prototype presented in this thesis.
The tool is build as an bokeh application that fetches the data from an SQL data base.
Why an database, why an SQL database. -> databases are a stable of datascience for a reason. SQL bring many advantages over just storing everything as plain text. easy to maintain, data easily accessible for other purposes.
Why bokeh. This choice is less obvious and offers more possible options. For this prototype bokeh offers the optional ready to use package, integrated server, visualization using the Javascript, HTML as CSS, powerful dataprocessign tools of python. Abstracted to an level that makes it easy to use while still providing access for low level custumizations.

At this moment the tool provides all the nessacary critical first insights into loci of interest.
Next steps are split into first of course deployment (utilize that it is so easy to do with bokeh) of the tool to make the tool accessible. Further implementation of small features that improve usability.

\section{Overlap of CAD Associated Variants with Regulatory Elements is Enriched in Heart, Artery \& Lung Tissue}
The database makes all the collected not only easily accessible for visualization purposes, but also for follow up studies.
Data for the datas sake is not useful. For this reason we are utlizing the data for an initial follow up study, looking for biosamples with \acp{cCRE} enriched in \ac{cad} \ac{gwas} variants via Fisher's exact test. This way identifying 34 biosamples (of [NUMBER] tested) that show significant overrepresentation (figure \ref{} \& table \ref{}). After further annotation of these biosample, over 40\,\% (14/34) of these biosamples stem from heart or artery tissue and are therefore directly affected by artheriosclerosis. An additional 20\,\% stem from lung tissue. Interesting among the other identified tissues are two hits for immune cells, which more specifically are annotated as CD14+ monocytes. \\
The same method could be applied to check for the overlap of disease associated variants with the enhancers identified as part of the \ac{abc} model. Further using the enhancer-promotor connections, to identify potentially affected genes.
