\section{Coronary artery disease}
\label{sec:cad}
- CAD is serious.
- Describe the pheontype
- Describe prevalence
- Check intro of Anja, papers for more info
- Go into treatment and Risk
- herditary parts in disease

\section{GWAS}
\label{sec:gwas}

    \subsection{GWAS}
    \label{subsec:gwas_general}
    What are GWAS. Describe how it works, why we do it.

    \subsection{Post GWAS}
    \label{subsec:gwas_limit}
    And their limitations. Possible follow up studies. Focus on computational and cell based assays.

\section{Muscle Cells in CAD}
\label{sec:haosms}
- We now that smooth muscle cells play a key role
- It is widely accepted that there is not only one type of smooth muscle cell
- Go into the contractile phenotype and synthetic phenotype
    - TGFb or PDGF used to induce them
    - contractile phenotype is thought to be protective

\section{PDGF Signaling and Oxidative Stress}
\label{sec:haosms}
    \subsection{PDGF Signaling}
    \label{subsec:pdf_signaling}

    \subsection{ROS}
    \label{subsec:ROS}
    - ROS in general and the role of ROS in disease

    \subsection{ROS in PDGF Signaling}
    \label{subsec:ROS_signaling}
    - ROS as a second messenger in PDGF signaling

\section{Complementary High Through Put Methods}
\label{sec:bioinformatics}

    \subsection{Linkage Disequilibrium}
    Linkage disequilibrium is a parameter from populations genetics that describes the non-random association of two or more allels. The LD is often quantified using the correlation coefficient $r^2$ \cite{slatkinLinkageDisequilibriumUnderstanding2008}.

    $$ D_{AB} = p_{AB} − p_A p_B $$
    $$ r^2 = \frac{D_{AB}^2}{p_A (1-p_A) \times p_B (1-p_B)} $$

    Where $p_A$ and $p_B$ is the frequency of the alleles A and B respectively. $p_{AB}$ is the frequency of the AB haplotype.

    \subsection{Locus To Gene Scores}

    \subsection{Regulatory Build}
    The ensembl regulatory build compiles a summary of regulatory regions found in the human genome. It is build on the basis publically available epigenetic marks and transcription factor binding and contains Promotors, Proximal enhancers, distal enhancers and CTCF binding sites \cite{zerbinoEnsemblRegulatoryBuild2015}.

    \subsection{ENCODE cCREs}

    \subsection{scATAC Seq}

    \subsection{ABC Model}

    \subsection{TADs}

\section{Aim of the thesis}
\label{sec:Aim}
- Build tool for visual exploration of the CAD GWAS data.
- Establish a system to test the role of ROS in CAD.
