\section{Coronary artery disease}
\label{sec:cad}
\Ac{cad} is among the leading causes of death in the (western) worldwide, being prevalent in about 6.7 \% of American adults and killing more than 350'000 people in the USA in 2019 alone \cite{centersfordiseasecontrolandpreventionHeartDiseaseFacts2022, fryarPrevalenceUncontrolledRisk2012}. \Ac{cad} is characterized by the build up of fatty plaques in the arteries leading to the heart. This process, also called atherosclerosis, can interrupt the blood supply to the heart \cite{nationalhealthserviceHeartAttack2017}. Its most common complication is \ac{mi}, which usually manifests as chest pain (angina) and can cause serious damage to the heart muscle.  Next to common and well known life-style factors like tobacco use or physical inactivity, \ac{cad} risk additionally has a hereditary component \cite{taskforcemembers2013ESCGuidelines2013}.


\section{Muscle Cells in CAD}
\label{sec:haosms}
The lumen of a typical blood vessel is surrounded by three distinct layers. The outer adventitia is rich in connective tissue, shapes the vessel and wraps the media. The media contains \acp{vsmc}. \Acp{vsmc} of the media are required to mediate vasodilation and vasoconstriction according to signaled requirements. The inner layer consists of endothelial cells that define the lumen of the vessel. \cite{tuckerAnatomyBloodVessels2022a, yapSixShadesVascular2021}

For the longest time, the role of \acp{vsmc} in the development and progression of atherosclerosis has been underestimated and over simplified. They have simply been considered to be either promoting of arteriosclerosis or beneficial for plaque stability. Only with the emergence of novel and exciting technologies like single cell transcriptomics and lineage tracking, this view is changing to a more differentiated one. \cite{grootaertVascularSmoothMuscle2021, yapSixShadesVascular2021} The study of \acp{vsmc} in arteriosclerosis is rapidly evolving, and the underlying models being adjusted accordingly. The black and white idea of \acp{vsmc} in arteriosclerosis existing either as a differentiated (contractile) phenotype or as a dedifferentiated (synthetic) phenotype, is making place for a model that considers a diverse set of dedifferentiated phenotypes \cite{grootaertVascularSmoothMuscle2021, yapSixShadesVascular2021}. The phenotypic switch describes the down regulation of contractile markers and can give rise to a diverse bouquet of different phenotypes which can be found in the fibrous cap and plaque core \cite{grootaertVascularSmoothMuscle2021}. Thier number as well as their impact on disease progression are still the subject of intensive research.

Two external stimuli that seem to play central roles as cytokines determining the fate of \acp{vsmc} in artherogenesis are \ac{tgf} \& \ac{pdgf}.


\section{\ac{tgf} Signaling}
\label{sec:tgf}

    \subsection{\ac{tgf} Signaling in General}
    \label{subsec:tgf_the_cytokine}
    \ac{tgf} is a summarizing term for a super family of cytokines, the most prominent of which is \ac{tgf}1. After secretion and activation, the active \ac{tgf} dimer binds to a heteromeric receptor complex. The intracellular signaling is mainly implemented via Smad transcription factors. The effects of \ac{tgf} is highly dependent on the cell type and can even be pleiotropic for cells of the same type. The most prominent function of \ac{tgf} is its role in anti-inflammatory regulation of immune cells. \cite{goumansTGFvSignalingControl2018, batlleTransformingGrowthFactorv2019}

    \subsection{\ac{tgf} Signaling in VSMCs \& atherosclerosis}
    \label{subsec:pdf_signaling}
    In the context of \acp{vsmc}, \ac{tgf} promotes proliferation and hypertrophy. Further, it promotes \ac{vsmc} differentiation, via elevation of contractile gene expression as well as the down regulation of \ac{klf4} \cite{davis-dusenberyDownregulationKruppellikeFactor42011}, a \ac{tf} prominent for its application in inducing pluripotency \cite{takahashiInductionPluripotentStem2007} that is also required for phenotype switching. This way hindering \cite{davis-dusenberyDownregulationKruppellikeFactor42011} or potentially reversing phenotype switching \cite{panSingleCellGenomicsReveals2020}.

\section{PDGF Signaling}
\label{sec:pdgf}
    \subsection{PDGF Signaling in General}
    \label{subsec:pdgf_the_cytokine}
    Five different PDGF isoforms have been identified that form as dimeric combination of four distinct polypeptide chains (PDGF-AA, PDGF-AB, PDGF-BB, PDGF-CC \& PDGF-DD). All five isoforms bind to tyrosine kinase receptors (\ac{pdgfr}\alpha \& \ac{pdgfr}\beta). Upon activation, the receptor dimerizes, allowing autophosphorylation which activates the kinase domain and creates binding sites for signaling molecules. The active receptor is involved in a plethora of prominent messaging pathways like the \ac{map}-kinase pathway, \ac{PI3K}-signaling or \ac{stat}-signaling. All these pathways are ultimately involved in the promotion of cellular proliferation, survival and migration \cite{chenPlateletderivedGrowthFactors2013, heldinTargetingPDGFSignaling2013, huTargetingPlateletderivedGrowth2015}.

    The predominantly expressed by endothelial cells seems to be \ac{pdgf} \cite{andraeRolePlateletderivedGrowth2008, heldinTargetingPDGFSignaling2013} which acts as a paracrine activator for \acp{vsmc} and other mesenchymal cells \cite{heldinTargetingPDGFSignaling2013}. Signaling via \ac{pdgf} and the \ac{pdgfr}\beta plays an important role in development of multiple tissues, e.g. in the development of the cardio vascular system \cite{leveenMiceDeficientPDGF1994}. After completed development, \ac{pdgf} picks up an important role in wound healing processes \cite{robsonPlateletderivedGrowthFactor1992}. The role of \ac{pdgfr}\beta signaling in pathologic processes like cancer or cardio vascular disease has been a subject of extensive study for decades \cite{heldinTargetingPDGFSignaling2013, rainesPDGFCardiovascularDisease2004}.

    \subsection{PDGF Signaling in VSMCs \& atherosclerosis}
    \label{subsec:pdgf_in_disease}
    In the context of \acp{vsmc}, \ac{pdgf} was shown to increase \ac{klf4} levels, which results in up-regulation of mesenchymal markers as well as the loss of contractile markers. Ultimately, serving as an external stimulus for proliferation and phenotype switching \cite{yapSixShadesVascular2021}.

    Similarly to the overall role of \acp{vsmc} in arteriosclerosis, the role of \ac{pdgf} is still the subject of extensive study. All PDGF isoforms are abundantly found in arteriosclerotic cell walls, and further \ac{pdgfr} expression is elevated in affected vessels \cite{huTargetingPlateletderivedGrowth2015}. For a long time PDGF signaling and inflammation has been assumed to be disease promoting \cite{andraeRolePlateletderivedGrowth2008, chenPlateletderivedGrowthFactors2013, hePDGFRvSignallingRegulates2015, huTargetingPlateletderivedGrowth2015} and in 2015 \textcite{hePDGFRvSignallingRegulates2015} showed that \ac{pdgfr}\beta signaling in mouse model leads to inflammation and increased plaque formation. In contrast to this consensus, \textcite{newmanMultipleCellTypes2021} were recently able to demonstrate, that sustained signaling via \ac{pdgfr}\beta is required for \ac{vsmc} involvement. They Further observed, again in mouse model, that lack of \ac{vsmc} involvement during plaque formation, can be temporarily compensated by non-\ac{vsmc}-derived cells, but long-term leads to instability of arteriosclerotic lesions.

    \subsection{ROS in PDGF Signaling}
    \label{subsec:ROS_signaling}
    \ac{ros} is a broad term for a class of highly reactive molecules derived from \ac{o2}. They are traditionally infamous for the damage they can do to proteins and nucleic acids when not kept in check, potentially causing irreparable damage leading to cell death. Recently, this perception has been shifting, and specially \ac{h2o2} and \ac{o2-}, are being recognized for their role in cellular signaling. \cite{siesReactiveOxygenSpecies2020}

    Human cells contain dozens of enzymes, which are capable of generating \ac{ros} and enzymatically maintain a steady redox state \cite{siesReactiveOxygenSpecies2020}. \ac{h2o2} and \ac{o2-} serve as important second messengers in the central nervous system \cite{nayerniaNewInsightsNOX2014} or in the reapir of vascular lesions \cite{andraeRolePlateletderivedGrowth2008}. Interestingly, the generation of \ac{ros} as a second messenger gets triggered by stimulation with\ac{pdgf} \cite{sundaresanRequirementGenerationH2O21995, bouziguesRegulationROSResponse2014a}.


    \section{GWAS}
    \label{sec:gwas}
    The hereditary components of disease onset and progression can provide access to its pathology on a molecular level.

    \subsection{GWAS}
    \label{subsec:gwas_general}
    An amazing resource for getting a first glance into these interactions are \ac{gwas}, a method that allows for the identification of genetic variants associated with a phenotype.

    While \ac{gwas} were initially an extraordinary endeavor, requiring the evaluation of hundreds or thousands of participants, they have gotten a lot more accessible with the availability of genetic data from public biobanks. After profiling the cohort on a genomic level (usually via microarrays but \ac{wgs} is probably the future) and phenotypically, the collected data needs to pass through several steps of quality control, e.g. for the removal of rare variants, miss matched phenotypes, etc. Afterwards, variants which were not directly analyzed are inferred from a reference. The final step of the initial analysis is the statistical mode, where a regression model is used to test for association of all variants with the phenotype in question. It is crucial to be completely aware of potential biases, that might have introduced in this process, some of which (like  age, sex or ancestry) can and need to be included as covariant in the used model. \cite{uffelmannGenomewideAssociationStudies2021, flintGWAS2013} The model will output a list of p-values, effect sizes (and their direction) for all tested variants. A \ac{gwas} is the first important step in determining causal variants for disease and therefore a first glimpse into the molecular biology of the observed phenotype \cite{uffelmannGenomewideAssociationStudies2021}.

    \subsection{postGWAS}
    \label{subsec:gwas_limit}
    Unfortunately, \ac{gwas} are just the first step in a longer journey of establishing causal loci to gene links, uncovering the molecular basis of disease, and implementing tools for clinical risk prediction. A plethora of follow-up analyses (postGWAS) can and need to be performed to determine a set of credible variants and to assess their molecular mechanism.

    The first important follow-up, that is usually done immediately, is fine-mapping. Due to the complex \ac{ld} of variants in the human genome (see sec. \ref{subsec:ld}), identified loci in \ac{gwas} unusually do not contain a single significant variant, but are made up of a potentially large set of linked variants. Fine-mapping describes the process of identifying the actually causal variant in this mess. Multiple very sophisticated statistical methods have been developed, the most popular of which is Bayesian modelling, which yields variant specific \acp{pip} that form a credible set of potentially causal variants. It is important to remember, that fine-mapping is not a solved problem, available methods are continuously improving and will most likely keep getting more complex with increasing complexity of the studied phenotypes. Further fine-mapping is a statistical approach which will never be able to determine causality! \cite{schaidGenomewideAssociationsCandidate2018, uffelmannGenomewideAssociationStudies2021}

    After the identification of likely causal variants, the next steps aim to gain information on their effect in determining the analyzed phenotype. Variants still require mapping to impacted genes, associated pathways and relevant tissues to get a glance of the complete image. For these steps no standard protocols exists and the procedure highly depends on the genomic context of the variant. Coding variants are and offer themselves to be immediately studied on a protein level, while non-coding variants are greatly benefit from the consultation of more high throughput data in the form of e.g. \acp{eqtl} \cite{uffelmannGenomewideAssociationStudies2021}.

    Finally, all the previous results can and need to be taken back to the wet lab, to verify and extend the ideas derived from statistical models. Utilizing all the recent great advances in molecular and cellular biology, such as the development of increasingly comprehensive \textit{in vitro} models as well and their manipulation via methods like \ac{crispr}-Cas gene-editing \cite{lichouFunctionalStudiesGWAS2020}.


    \section{Complementary High Through Put Methods}
    \label{sec:bioinformatics}
    The development of high trough put methods as well as the great increase in computing power over the last few years have spawned a plethora of incredible datasets that already have been and can be further utilized for post\ac{gwas} analyses. A short overview of some definitions and methods mentioned in this thesis can be found in the following paragraphs:

    \subsection{Linkage Disequilibrium}
    \label{subsec:ld}
    \Ac{ld} is a parameter from populations genetics that describes the non-random association of two or more alleles. The \ac{ld} is often quantified using the correlation coefficient $r^2$ \cite{slatkinLinkageDisequilibriumUnderstanding2008}.

        $$ D_{AB} = p_{AB} − p_A p_B $$
        $$ r^2 = \frac{D_{AB}^2}{p_A (1-p_A) \times p_B (1-p_B)} $$

    Where $p_A$ and $p_B$ is the frequency of the alleles A and B respectively. $p_{AB}$ is the frequency of the AB haplotype.

    The \ac{ld} becomes important in the context of \ac{gwas} because identified SNPs often do not occur in isolation, but a network of linked and significant variants can span large haplotype blocks in the genome \cite{slatkinLinkageDisequilibriumUnderstanding2008}.

    \subsection{Locus To Gene Scores}
    Problems of interpretation of \ac{gwas} data are already described in section \ref{sec:gwas}. \ac{l2g} scores are an attempt at overcoming the challenges of establishing causal relationships between variants and genes. The authors employed a machine learning-model to integrate fine-mapping with functional genomics data and \textit{in silico} predictions to link \ac{gwas} loci to their target genes. The output \ac{l2g} scores are calibrated to represent the probability (0, 1). \cite{mountjoyOpenApproachSystematically2021}

    \subsection{Regulatory Build}
    The ensembl regulatory build compiles a summary of putative regulatory regions found in the human genome. It is constructed from publically available data on epigenetic marks and \ac{tf} binding sides. It contains promotors, proximal enhancers, distal enhancers and \ac{ctcf} binding sites. \cite{zerbinoEnsemblRegulatoryBuild2015}

    \subsection{ENCODE \acp{cCRE}}
    Very similarly, the \ac{encode} project summarizes \ac{dna} accessibility and chromatin modification data into \acp{cCRE}. Regions showing high DNase signal are further annotated to be \ac{pELS} or \ac{dELS}, \ac{PLS}, other regions with high \ac{H3K4me3} signal (which might represent poised or non-canonical promotors), or \ac{ctcf}-only elements based on the existence of \ac{H3K4me3}, \ac{H3K27ac} and \ac{ctcf} signals.
    \cite{mooreExpandedEncyclopaediasDNA2020}

    \subsection{\acs{atac}}
    \acf{atac} is a method to access chromatin accessibility in the genome. \Ac{atac} utilizes the hyperactive Tn5 transponase to insert sequencing adapters into accessible regions of chromosome. \ac{dna} is purified and amplified via \ac{pcr} and then sequenced. Mapping sites with insertions to the genome allows for the identification of highly accessible genomic regions. \cite{buenrostroTranspositionNativeChromatin2013, buenrostroATACseqMethodAssaying2015}

    \ac{pcr} amplification of the \ac{dna} makes this method extremely sensitive. Pushing the requirement of biomaterial to the minimum, \ac{atac} is applicable on a single cell level. For sc\ac{atac}, individual cells are isolated and their \ac{dna} tagged with barcoded primers during the \ac{pcr}. These barcodes allow mapping of \ac{atac} data to the isolated cells. \cite{buenrostroSinglecellChromatinAccessibility2015}

    \subsection{\acs{abc} Model}
    The \acf{abc} model grants insights into potential cell specific enhancer-gene interactions based on chromatin state, outperforming previously used methods \cite{fulcoActivitybycontactModelEnhancer2019a, nasserGenomewideEnhancerMaps2021a}.

        $$ ABC\,score_{E, G} = \frac{A_E \times C_{E,G}}{\sum\limits_{all\,elements\,e\,within\,5\,Mb\,of\,G} A_e \times C_{e, G}} $$

    Generally speaking, the model incorporates the activity of an enhancer $A_E$ as well as contacts with the gene of interest $C_{E,G}$, normalized by the total effect of all elements in the area \cite{fulcoActivitybycontactModelEnhancer2019a, nasserGenomewideEnhancerMaps2021a}.

    \subsection{Hi-C \& \acsp{tad}}
    Hi-C is a method for mapping chromosomal conformation. For achieve this, genome associated proteins are cross-linked with formaldehyde, \ac{dna} is digested with restriction enzymes and generated overhangs are filled in with biotinylated nucleotides. The resulting fragments are ligated to covalently link \ac{dna} fragments, which were originally in close spatial proximity. The \ac{dna} is purified and fragmented, allowing the pulldown of fragments containing junctions sites via the filled in biotin tags. After sequencing of the enriched fragments, their sequences are mapped to the genome, identifying interacting DNA regions. \cite{lieberman-aidenComprehensiveMappingLongRange2009, witDecade3CTechnologies2012}

    Looking at Hi-C data, \acp{tad} were identified to be a basic feature of genome organization with an average size of 880 kb \cite{dixonTopologicalDomainsMammalian2012, wang3DGenomeBrowser2018}. What makes \acp{tad} of such high interest is the fact that interactions of \ac{dna} sequences are usually confined within \acp{tad}. Tissue-specific genes and their enhancers are usually found in the middle of \acp{tad}, while the edges enrich for housekeeping genes and \ac{ctcf} binding sides which might serves as insulators between different domains \cite{pomboThreedimensionalGenomeArchitecture2015}.


\section{Aim of the thesis}
\label{sec:Aim}
The aims of this thesis are split into two quite distinct projects that both ultimately aim to contribute to a better understanding of artheriosclerosis and \ac{cad}:

\begin{itemize}
    \item The split role of \ac{pdgf} during progression of artheriosclerosis (see section \ref{subsec:pdgf_in_disease}), indicates that \ac{pdgf} signaling is neither completely benefitial nor disadvantageous to diseases but there is an optimal dose. Combining this theory with the fact that \ac{ROS} are involved in \ac{pdgf} signaling and also highly associated with artheriosclerosis \cite{burtenshawReactiveOxygenSpecies2019}, we hypothesized, that \ac{pdgf} signaling may cause oxidative stress, this way contributing to disease progression. The first part of this thesis will deal with the \textit{in vitro} characterization of \ac{pdgf} stimulated \acp{vsmc} and the establishment of a robust assay for oxidative stress in \acp{vsmc}.
    \item The summary statistics from \textcite{aragamDiscoverySystematicCharacterization2021} are a great resource. To make this data more accessible we are building an interactive web-based visualization tool, co-visualizing the data with different kinds of annotations like gene products, associated phenotypes from other \ac{gwas} or putative regulatory elements.
    \item Finally, we'll use the data curated for the visualization tool for postGWAS studies.
\end{itemize}

Have fun with my thesis, this still is a mess...
