\section{Coronary artery disease}
\label{sec:cad}
\Ac{cad} is one of the leading causes of death in western societies, demonstrated by a prevalence in 6.7\,\% of American adults and leading to the annual death of 350,000 people in the USA in 2019 \cite{centersfordiseasecontrolandpreventionHeartDiseaseFacts2022, fryarPrevalenceUncontrolledRisk2012}. \Ac{cad} is characterized by the build-up of fatty plaques in arteries supplying the heart with oxygen. This process, called atherosclerosis, can interrupt the blood supply to the heart \cite{nationalhealthserviceHeartAttack2017}. Its most common complication is \ac{mi} which usually manifests as chest pain (angina) and may cause severe damage to the heart muscle. Long time, \ac{cad} can lead to \ac{hf}, the heart's inability to pump blood properly. Next to common and well-known lifestyle factors like tobacco use or physical inactivity, \ac{cad} risk has a hereditary component \cite{montalescot2013ESCGuidelines2013}.


\section{VSMCs in CAD}
\label{sec:haosms}
A typical blood vessel is constructed from three distinct layers surrounding the lumen. The outer adventitia is rich in connective tissue and shapes the vessel. It wraps the media, the middle layer containing \acp{vsmc}. \acp{vsmc} are required to mediate vasodilation and vasoconstriction according to signaled requirements. The inner layer consists of endothelial cells that define the vessel's lumen. \cite{tuckerAnatomyBloodVessels2022a, yapSixShadesVascular2021}\\
Cell types commonly associated with atherogenesis are endothelial cells, immune cells, and \acp{vsmc} \cite{tabasRecentInsightsCellular2015}. For a long time, the role of \acp{vsmc} in the development and progression of atherosclerosis has been underestimated and over-simplified. \acp{vsmc} have been considered either to be promotive of atherosclerosis progression or beneficial for plaque stability. Only with the emergence of novel and exciting technologies like \ac{sc} transcriptomics and lineage tracking has this model changed into a more multifaceted one. \cite{liuSmoothMuscleCell2019, grootaertVascularSmoothMuscle2021, yapSixShadesVascular2021} The study of \acp{vsmc} in atherosclerosis is rapidly evolving, and the underlying models are being adjusted accordingly. The black and white idea of \acp{vsmc} in atherosclerosis existing either as a differentiated (contractile) phenotype or as a dedifferentiated (synthetic) phenotype is shifting towards the consideration of a diverse set of dedifferentiated phenotypes \cite{liuSmoothMuscleCell2019, grootaertVascularSmoothMuscle2021, yapSixShadesVascular2021}. The phenotypic switch describes the loss of contractile markers and can give rise to a diverse bouquet of different phenotypes, which can be found in the fibrous cap and plaque core \cite{grootaertVascularSmoothMuscle2021}. The characterization of these dedifferentiated phenotypes and their impact on disease progression is still the subject of intensive research.\\
Among others, two external stimuli that seem to play central roles as cytokines determining the fate of \acp{vsmc} in atherogenesis are \acf{tgf} and \acf{pdgf}.


\section{\acs{tgf} signaling}
\label{sec:tgf}

    \subsection{\ac{tgf} signaling in general}
    \label{subsec:tgf_the_cytokine}
    The term \acf{tgf} describes a superfamily of cytokines, the most prominent of which is \ac{tgf}1. After secretion and activation, the active \ac{tgf} dimer binds to a heteromeric receptor complex. The intracellular signaling is mainly implemented via Smad transcription factors. The cellular effects of \ac{tgf} are highly dependent on the cell type and can even be pleiotropic for cells of the same type. The most prominent function of \ac{tgf} is its role in the anti-inflammatory regulation of immune cells. \cite{goumansTGFbetaSignalingControl2018, batlleTransformingGrowthFactorbeta2019}

    \subsection{\ac{tgf} signaling in VSMCs and atherosclerosis}
    \label{subsec:pdf_signaling}
    In the context of \acp{vsmc}, \ac{tgf} promotes proliferation and hypertrophy. Further, it promotes \ac{vsmc} differentiation via elevated gene expression of contractile marker genes. Additionally, \ac{tgf} mediates the decreased expression of \ac{klf4} \cite{davis-dusenberyDownregulationKruppellikeFactor42011}, a \ac{tf} prominent for its application in inducing pluripotency \cite{takahashiInductionPluripotentStem2007}. This way, \ac{tgf} hinders \cite{davis-dusenberyDownregulationKruppellikeFactor42011} or potentially reverses phenotype switching \cite{panSingleCellGenomicsReveals2020}.

\section{\acs{pdgf_simple} signaling}
\label{sec:pdgf}
    \subsection{\ac{pdgf_simple} signaling in general}
    \label{subsec:pdgf_the_cytokine}
    Five \ac{pdgf_simple} isoforms (\ac{pdgf_simple}-AA, \ac{pdgf_simple}-AB, \ac{pdgf}, \ac{pdgf_simple}-CC, and \ac{pdgf_simple}-DD) have been identified as dimeric combinations of four distinct polypeptide chains. All five isoforms bind to tyrosine kinase receptors. Upon activation, the \ac{pdgfr} dimerizes, allowing autophosphorylation, which activates the kinase domain and creates binding sites for signaling molecules. The active receptor is involved in a plethora of prominent messaging pathways like the \ac{map}-kinase pathway, \ac{PI3K}-signaling or \ac{stat}-signaling. These pathways ultimately promote cellular proliferation, survival, and migration \cite{chenPlateletderivedGrowthFactors2013, heldinTargetingPDGFSignaling2013, huTargetingPlateletderivedGrowth2015}.\\
    The \ac{pdgf_simple} isoform predominantly expressed by endothelial cells has been demonstrated to be \ac{pdgf} \cite{andraeRolePlateletderivedGrowth2008, heldinTargetingPDGFSignaling2013} which acts as a paracrine activator for \acp{vsmc} and other mesenchymal cells \cite{heldinTargetingPDGFSignaling2013}. \ac{pdgf} predominantly binds to \ac{pdgfr}\beta~and plays an essential role in the development of multiple tissues, amongst others, in the development of the cardiovascular system \cite{leveenMiceDeficientPDGF1994}. In adults, \ac{pdgf} picks up an function in wound healing processes \cite{robsonPlateletderivedGrowthFactor1992}. The contribution of \ac{pdgfr}\beta~signaling in pathologic processes like cancer or cardiovascular disease has been a subject of thorough study for decades \cite{heldinTargetingPDGFSignaling2013, rainesPDGFCardiovascularDisease2004}.

    \subsection{\ac{pdgf_simple} signaling in \acp{vsmc} and atherosclerosis}
    \label{subsec:pdgf_in_disease}
    Similarly to the overall role of \acp{vsmc} in atherosclerosis, the role of \ac{pdgf} is still the subject of intensive study. In the context of \acp{vsmc}, \ac{pdgf} has been shown to increase \ac{klf4} levels, which results in an increased expression of mesenchymal markers accompanied by the loss of contractile markers, serving as an external stimulus for proliferation and phenotype switching \cite{yapSixShadesVascular2021}.\\
    All \ac{pdgf_simple} isoforms are found in the cell wall of atherosclerotic vessels, and the expression of \ac{pdgfr} is elevated in affected vessels \cite{huTargetingPlateletderivedGrowth2015}. For a long time, \ac{pdgf_simple} signaling and inflammation have been assumed to be promotive of disease progression \cite{andraeRolePlateletderivedGrowth2008, chenPlateletderivedGrowthFactors2013, huTargetingPlateletderivedGrowth2015}, and recently \textcite{hePDGFRbetaSignallingRegulates2015} showed that \ac{pdgfr}\beta~signaling in a mouse model leads to inflammation and increased plaque formation. In contrast to this consensus, \textcite{newmanMultipleCellTypes2021} recently demonstrated that sustained signaling via \ac{pdgfr}\beta~is required for \ac{vsmc} involvement in atherosclerotic lesions and the construction of the fibrous cap. Their mouse model shows that the lack of \ac{vsmc} involvement during plaque formation can be temporarily compensated by non-\ac{vsmc}-derived cells. However, long term, the lack of \ac{vsmc} involvement leads to instability of atherosclerotic lesions.

    \subsection{ROS in PDGF signaling}
    \label{subsec:ROS_signaling}
    \Ac{ros} are a class of highly reactive molecules derived from \ac{o2}. They are traditionally infamous for their damaging effect on proteins and nucleic acids, potentially causing irreparable damage and ultimately leading to cell death. Recently, this perception has been shifting, especially \ac{h2o2} and \ac{o2-} are recognized for their role in cellular signaling. \cite{siesReactiveOxygenSpecies2020} \ac{ros} as intracellular messengers predominantly implement their effect by oxidation protein targets \cite{zeidaCatalysisPeroxideReduction2019}. \\
    Human cells contain dozens of proteins capable of generating \ac{ros} and enzymatically maintain a redox steady-state \cite{siesReactiveOxygenSpecies2020}. \ac{h2o2} and \acp{o2-} serve as important secondary messengers in the central nervous system \cite{nayerniaNewInsightsNOX2014}, for the repair of vascular lesions \cite{andraeRolePlateletderivedGrowth2008}, and \ac{pdgf} signaling \cite{sundaresanRequirementGenerationH2O21995, bouziguesRegulationROSResponse2014a}.


    \section{GWAS}
    \label{sec:gwas}

    \subsection{GWAS}
    \label{subsec:gwas_general}
    A fantastic resource for obtaining a first glance at these interactions are \acp{gwas}, a method that enables the identification of genetic variants associated with a phenotype.\\
    While \acp{gwas} were initially an extraordinary endeavor, requiring the evaluation of hundreds or thousands of participants, they have gotten a lot more accessible with the availability of genetic data from public biobanks. The first step is profiling a suitable cohort on a genomic level and its phenotypical characterization. Subsequently, the collected data must pass through several quality control steps to remove rare variants, mismatched phenotypes, and others. Afterward, not directly analyzed variants are inferred from a reference genome. The final step of the initial analysis is implementing a statistical model; a regression model is used to test for the association of all variants with the phenotype in question. It is crucial to be completely aware of potential biases, some of which (like age, sex, or ancestry) can and need to be included as covariants in the used model. \cite{uffelmannGenomewideAssociationStudies2021, flintGWAS2013} The model will output a list of p-values, effect sizes, and their direction for all tested variants. A \ac{gwas} is the first important step in determining causal variants for a phenotype and, therefore, a glimpse into its molecular basis \cite{uffelmannGenomewideAssociationStudies2021}.

    \subsection{postGWAS}
    \label{subsec:gwas_limit}
    Unfortunately, \acp{gwas} are just the first step in a long journey of establishing causal loci to gene links, uncovering the molecular basis of disease, implementing tools for clinical risk prediction, and developing treatment options. A plethora of follow-up analyses (post\acp{gwas}) are essential to convert the first list of associated variants into a set of credible variants and to assess the underlying molecular mechanisms.\\
    The first necessary follow-up is fine-mapping. Due to the complex \ac{ld} of variants in the human genome (see section \ref{subsec:ld}), loci identified in the \ac{gwas} usually do not contain a single variant. Instead, multiple variants in the vicinity may form a region of linked and significant variants. Fine-mapping identifies the actual causal variant(s) in this potentially complex cluster. Multiple sophisticated statistical methods have been developed. The popular approach of Bayesian modeling outputs variant-specific \acp{pip} that form a credible set of potentially causal variants. Noteworthy, fine-mapping methods are continuously refined and evolve alongside the increasing complexity of the studied phenotypes. Finally, fine-mapping is a purely statistical approach that can only identify potential causal variants which need to be confirmed via complementary approaches. \cite{schaidGenomewideAssociationsCandidate2018, uffelmannGenomewideAssociationStudies2021}\\
    After fine-mapping, the following steps aim to gain information on the causal variants' effect in determining the analyzed phenotype. Variants require mapping to impacted genes, associated pathways, and relevant tissues and cell types, providing helpful insight into the complete picture. For these steps, no standard protocols exist, and the procedure highly depends on the genomic context of the variant of interest. Coding variants are rare but may be immediately studied on a protein level. On the other hand, non-coding variants usually greatly benefit from the consultation of more high throughput data in the form of, e.g., \ac{eqtl} data \cite{uffelmannGenomewideAssociationStudies2021}.\\
    Finally, the results and ideas derived from statistical models can and need to be taken back to the wet lab to be extended and verified. Utilizing the recent remarkable advances in molecular and cellular biology, such as the development of increasingly comprehensive \textit{in vitro} models and their manipulation via methods like \ac{crispr}-Cas gene-editing \cite{lichouFunctionalStudiesGWAS2020}.


    \section{Complementary high through put methods}
    \label{sec:bioinformatics}
    The development of high trough put methods combined with the significant increase in computational power over the last few years have paved the way for post\ac{gwas}. A short overview of some definitions and methods mentioned in this thesis can be found in the following paragraphs:

    \subsection{Linkage disequilibrium}
    \label{subsec:ld}
    \Ac{ld} is a parameter used in population genetics that describes the non-random association of two or more alleles. The \ac{ld} is often quantified using the correlation coefficient $r^2$ \cite{slatkinLinkageDisequilibriumUnderstanding2008}.

        $$ D_{AB} = p_{AB} − p_A p_B $$
        $$ r^2 = \frac{D_{AB}^2}{p_A (1-p_A) \times p_B (1-p_B)} $$

    Where $p_A$ and $p_B$ are the frequency of the alleles A and B, respectively, and $p_{AB}$ is the frequency of the AB haplotype.\\
    The \ac{ld} becomes vital in the context of \acp{gwas} because identified \acp{snp} often do not occur in isolation but as a network of linked and significant variants that can span large haplotype blocks in the genome \cite{slatkinLinkageDisequilibriumUnderstanding2008}.

    \subsection{Locus to gene scores}
    The interpretation of GWAS data is prone to limitations and problems described in section \ref{sec:gwas}. \Ac{l2g} scores attempt to overcome the challenges of establishing causal relationships between variants and genes. The authors employed a machine learning model to integrate fine-mapping with functional genomics data and \textit{in silico} predictions to link \ac{gwas} loci to their target genes. The computed \ac{l2g} scores are calibrated to represent the probability (0,\,1). \cite{mountjoyOpenApproachSystematically2021}

    \subsection{Ensembl regulatory build}
    The Ensembl Regulatory Build compiles a summary of putative regulatory regions found in the (human) genome. It is constructed from publically available data on epigenetic marks and \ac{tf} binding sides. The build considers promotors, proximal enhancers, distal enhancers, and \ac{ctcf} binding sites. \cite{zerbinoEnsemblRegulatoryBuild2015}

    \subsection{ENCODE \acs{cCRE}}
    Similarly, the \ac{encode} summarizes \ac{dna} accessibility and chromatin modification data into \acfp{cCRE}. Based on the existence of \ac{H3K4me3}, \ac{H3K27ac}, or \ac{ctcf} marks, regions showing high DNase signal are further annotated to be \cite{mooreExpandedEncyclopaediasDNA2020}:
    \begin{itemize}
        \item \ac{pELS}
        \item \ac{dELS}
        \item \ac{PLS}
        \item Regions with high \ac{H3K4me3} signal which might represent poised or non-canonical promotors
        \item \ac{ctcf}-only elements
    \end{itemize}

    \subsection{\acs{abc} model}
    The \acf{abc} model grants insights into potential cell-specific enhancer-gene interactions based on chromatin state, outperforming previously used methods \cite{fulcoActivitybycontactModelEnhancer2019a, nasserGenomewideEnhancerMaps2021a}.

        $$ ABC\,score_{E, G} = \frac{A_E \times C_{E,G}}{\sum\limits_{all\,elements\,e\,within\,5\,Mb\,of\,G} A_e \times C_{e, G}} $$

    Generally speaking, the model incorporates the activity of an enhancer $A_E$ and contacts with the gene of interest $C_{E,G}$, normalized by the total effect of all elements in proximity \cite{fulcoActivitybycontactModelEnhancer2019a, nasserGenomewideEnhancerMaps2021a}.

    \subsection{\acs{atac}}
    \Acf{atac} is a method to study chromatin accessibility in the genome. \Ac{atac} utilizes the hyperactive Tn5 transponase to insert sequencing adapters into accessible chromatin regions. \ac{dna} is purified and amplified via \ac{pcr} and then sequenced. Mapping sites with insertions in the genome allows for identifying highly accessible genomic regions. \cite{buenrostroTranspositionNativeChromatin2013, buenrostroATACseqMethodAssaying2015}\\
    Employing \ac{pcr} amplification renders \ac{atac} an extremely sensitive method. Pushing the requirement of biomaterial to the minimum, \ac{atac} is applicable on a single-cell level. For sc\ac{atac}, individual cells are isolated, and their \ac{dna} is tagged with barcoded primers during the \ac{pcr}. These barcodes allow mapping of \ac{atac} data to the isolated cells. \cite{buenrostroSinglecellChromatinAccessibility2015}

    \subsection{Hi-C and \acsp{tad}}
    Hi-C is a method for mapping chromosomal conformation. Genome-associated proteins are cross-linked with formaldehyde, the \ac{dna} is digested with restriction enzymes, and generated overhangs are filled-in with biotinylated nucleotides. The resulting fragments are ligated, covalently linking \ac{dna} fragments initially in close spatial proximity. The \ac{dna} is purified and fragmented, allowing the pulldown of fragments containing junctions sites via the filled-in biotin tags. After sequencing the enriched fragments, their sequences are mapped to the genome, identifying interacting \ac{dna} regions. \cite{lieberman-aidenComprehensiveMappingLongRange2009, witDecade3CTechnologies2012}\\
    Looking at Hi-C data, \acp{tad} were identified as a fundamental feature of genome organization with an average size of 880 kb \cite{dixonTopologicalDomainsMammalian2012, wang3DGenomeBrowser2018}. What makes \acp{tad} interesting is that interactions of \ac{dna} sequences are usually confined within \acp{tad}. Tissue-specific genes and their enhancers are usually found in the middle of \acp{tad}. At the same time, the edges enrich housekeeping genes and \ac{ctcf} binding sides, which might serve as insulators between different domains \cite{pomboThreedimensionalGenomeArchitecture2015}.


\section{Aim of the thesis}
\label{sec:Aim}
This thesis is split into two projects that both ultimately aim to contribute to a refined understanding of atherosclerosis and \ac{cad}:

\begin{itemize}
    \item The split role of \ac{pdgf} during the progression of atherosclerosis (see section \ref{subsec:pdgf_in_disease}) indicates that \ac{pdgf} signaling is neither wholly beneficial nor disadvantageous to diseases, but there is an optimal stimulation dosage. Since \ac{ros} are mutually involved in \ac{pdgf} signaling \cite{sundaresanRequirementGenerationH2O21995, bouziguesRegulationROSResponse2014a} and atherosclerosis \cite{burtenshawReactiveOxygenSpecies2019}, we hypothesize a functional linkage. \ac{pdgf} signaling may cause oxidative stress, this way contributing to disease progression. The first part of this thesis will address the \textit{in vitro} characterization of \ac{pdgf} stimulated \acp{vsmc} and the establishment of a robust assay for oxidative stress in \acp{vsmc}.
    \item High-throughput methods are a stable of modern bioscience and a great resource. The study by \textcite{aragamDiscoverySystematicCharacterization2021} provides a vast and important dataset for \ac{cad} research. One of the goals of this thesis is to make this data and the evaluation of its genomic context easily accessible to medical researchers in the form of an interactive web-based visualization tool. The \aca{gwas} Navigator will visualize \ac{gwas} summary statistics with different annotations in the form of associated phenotypes from other \acp{gwas} and putative regulatory elements.
    \item Finally, the data curated for the \aca{gwas} Navigator will be subject to an enrichment analysis, studying the overlap of disease-associated variants with regulatory elements in a diverse set of biosamples.
\end{itemize}
