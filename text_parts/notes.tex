
Here I'll discuss my results should I ever finish the rest of my thesis.



SOME THING FOR ROS:


    Recently, it was reported that STAT also has a cellular
    nongenomic function. STAT interacts with GRIM-19, a
    subunit of mitochondria respiratory complex I, which determines whether STAT3 is imported into mitochondria (12,13).
    Furthermore, there is direct evidence that STAT is present in
    the mitochondria of cultured cells and primary tissues. STAT
    mitochondrial importation selectively stabilizes and increases
    mitochondria respiratory complexes, allowing them to orchestrate responses to stimuli (14,15). As mitochondria respiratory
    complex I and III are thought to be the main source for ROS
    generation (16,17), it has been suggested that STAT1 facilitates
    ROS production and apoptosis \cite{leeRoleSTAT1IRF12007}. \cite{wangSTATROSCycleExtends2018}

    Based on our results, STAT1, which is primarily induced by IFN-γ, is a potent effector responsible for ROS production and loss of ΔΨm in LPS/D-GalNinduced fulminant liver injury, evidenced by the finding that ROS production and loss of ΔΨm were almost completely abrogated in STAT1(−/−) and IFN-γ(−/−) mice (Figure 2). -> INF induced ROS production and apoptosis.


SMC PHENOTYPES:


Contractile vs. synthetic VSMCs: loss of contracile marker snad synthetic, migratory pehnotype. Historically, this phenotypic switch was viewed as the hallmark of vascular repair. VSMCs retain their noncontractile status during atherosclerosis due to the continuous exposure to phenotypic switching-inducing stimuli

Foam cells: he majority of foam cells in atherosclerotic plaques appear derived from VSMCs. lipid loaded-VSMCs in culture secrete a variety of pro-inflammatory mediators48 and undergo apoptosis from free cholesterol overload,49 likely compromising plaque stability.

Macrophage-like: up-regulate ‘macrophage markers’, such as LGALS3, CD68,51 and pro-inflammatory cytokines. difficult to predict their impact on atherosclerotic plaque growth and stability. their pro-inflammatory profile, macrophage-like VSMCs are likely detrimental for plaque stability

Adipocyte-like: These studies emphasize again the high level of plasticity of VSMCs, but the role and abundance of adipocyte-like VSMCs in atherogenesis is unclear, illustrated by a postulated neutral-to-negative position on the ‘plaque-stability scale’.

Osteochondrogenic: Vascular calcification is a tightly regulated process, primarily driven by VSMCs developing an osteochondrogenic phenotype. In particular, micro-calcified deposits (<50 mm) are associated with increased inflammation and mostly observed in the fibrous cap of human lesions, and thus considered detrimental for plaque stability. In contrast, macrocalcifications (>200 mm) often accumulate in the deep intima or necrotic core in organized structures and may promote plaque stability.

Myofibroblast-like: loss of Tcf21 inhibits phenotypic modulation and attenuates fibromyocyte number in the fibrous cap, suggesting an athero-protective role (Figure 3).

EC-like: Direct evidence of endothelial-like VSMCs in human atherosclerosis is currently lacking.

MSC-like: The highly plastic nature of mesenchymal-like VSMCs offers the opportunity to therapeutically push these cells into a plaque-stabilising phenotype (Figure 3). The VSMCs undergo a ‘reprogramming process’ controlled by Kru ̈ppel-like factor 4 (KLF4) and can transdifferentiate into different cell types, including macrophage-like, EC-like, chondrocyte-like, and adipocyte-like under defined in vitro conditions.




Synthetic VSMCs play a major role in forming and maintaining the fibrous cap by secreting ECM, yet their overall role in plaque stability could be detrimental. Synthetic VSMCs secrete a wide variety of proinflammatory molecules and matrix-degrading enzymes that can cause cell death of neighbouring cells.42,114 Pro-inflammatory cytokines, such as IL1b, IL6, and MCP1 promote atherogenesis by stimulating monocyte recruitment and cell death119 and synthetic VSMCs express a range of adhesion molecules and receptors (e.g. Toll-like-receptors) that promote monocyte recruitment and regulate intracellular inflammatory signalling, respectively. Moreover, synthetic VSMCs secrete extracellular vesicles (EVs) that can drive vascular calcification (see Section 5.2.4).59 Hence, synthetic VSMCs are beneficial for fibrous cap formation, but depending on the local environment and the disease stage, may promote inflammation, calcification, cell senescence, and plaque instability.



The contractile phenotype is positively controlled by the growth factor TGFb.
PDGF-BB suppresses contractile gene expression via different mechanisms.
\cite{grootaertVascularSmoothMuscle2021}

Vascular smooth muscle cells (VSMCs) play a key role in atherogenesis and have historically been considered beneficial for plaque stability.

Thinning of the fibrous cap in advanced plaques increases the risk of rupture, which triggers thrombus formation and subsequent clinical complications including heart attack and stroke (Libby et al., 2011; Tabas et al., 2015).

The model is rapdily evolving and adjusted: it isnt even clear if cells migrate and proliferate or if they proliferate and migrate then.

The synthetic VSMC phenotype is characterised by loss of contractile marker expression and up‐regulation of selective gene sets, including pro‐inflammatory cytokines and MMPs, leading to increased cell migration, proliferation, and secretion of pro‐inflammatory cytokines.

Therefore, it has been proposed that VSMC‐derived cells can both improve plaque stability and exacerbate plaque rupture

To develop efficient therapeutic strategies to limit cardiovascular risk, additional knowledge about how specific VSMC‐derived cell types function in mature plaque is therefore needed. Additionally, mechanistic insight into the regulation of VSMC plasticity is required to enable specific interventions.
\cite{harmanRoleSmoothMuscle2019}

Up until recently, vSMCs were classified as either contractile or dedifferentiated (ie, synthetic).

The central dedifferentiated vSMC type that we classified is the mesenchymal-like phenotype.

The 3 main cellular layers forming the vessel wall are the adventitia, media, and the intima, surrounding the lumen. In the media, the middle layer, vascular smooth muscle cells (vSMCs) are the major cellular component. These vSMCs contribute to the integrity of the vessels and are able to adequately respond to stimuli of vasoconstriction and vasodilation

For decades, vSMC activation and dedifferentiation has been regarded as the adoption of a single synthetic, proliferative phenotype. However, as revealed by recent scRNA-seq analyses, the diversity of vSMC phenotypes is far more sophisticated.

The combination of scRNA-seq and lineage tracing is extremely useful as it allows in-depth vSMC phenotypic characterization.

Contractile vSMCs are regarded as differentiated and quiescent cells under physiological conditions, expressing a panel of typical contractile proteins that is crucial to maintain vascular tension. Contractile vSMCs exhibit an elongated, spindleshaped morphology and express a well-characterized set of contractile markers including smooth muscle actin (ACTA2), smooth muscle myosin heavy chain (MYH11), smooth muscle protein 22-alpha (SM22α/TAGLN), smoothelin (SMTN), and calponin (CNN1). Expression of these proteins is controlled by the transcription factors MYOCD (myocardin) and SRF (serum response factor), both of which are involved in the regulation of differentiation to contractile vSMCs.

In addition, external stimuli, such as TGF-β and heparin, play a pivotal role in promoting and maintaining the vSMCs contractile phenotype.

Once pathological processes in the vessel wall are initiated, vSMCs respond by changing their phenotype and function. A plethora of pathological cues induce these changes: factors from the circulation, compounds and proteins produced by activated endothelial cells, fibroblasts, perivascular adipocytes or inflammatory cells, (lack of) mechanical stress, damaged ECM protein fragments, or ECM-derived growth factors.

KLF4 is regulated by various signaling complexes at transcriptional and posttranslational levels.57,58 After exposure to PDGF-BB (platelet-derived growth factor BB), a stimulus of vSMC proliferation and phenotype switching, elevated levels of KLF4 were identified.

Induction of KLF4 in vSMCs results in a phenotypic switch from contractile to mesenchymal-like and initiates the expression of mesenchymal markers such as stem cell antigen-1 (SCA1)/LY6A, CD34, and CD44.40,41 During this transition, while gaining expression of mesenchymal markers, the contractile vSMCs lose expression of their contractile markers.
\cite{yapSixShadesVascular2021}

To prevent or reverse the phenotypic transition to mesenchymal-like vSMCs, the expression of KLF4 can be suppressed by TGF-β or miR-143/145 to maintain the contractile vSMC phenotype.

Alternatively, mesenchymal-like vSMC may undergo further changes into other vSMC phenotypes.31

In addition, we argued that the data point in the direction of the mesenchymal-like cells giving rise to the other vSMC phenotypes; however, further experimental validation is needed to fully support this hypothesis.
\cite{yapSixShadesVascular2021}

This includes the activation of SMC marker genes by TGFβ1 and contractile agonists such as angiotensin II as well as repression by PDGF BB

inflammatory cytokines such as IL-1β can induce rapid downregulation of expression of multiple SMC differentiation marker genes
\cite{alexanderEpigeneticControlSmooth2012a}

Using single-cell and bulk RNA-sequencing analyses of the brachiocephalic artery region and in vitro models, we provide evidence that SMC-to-MF transitions are induced by PDGF and transforming growth factor-β and dependent on aerobic glycolysis, while EndoMT is induced by interleukin-1β and transforming growth factor-β. Together, we provide evidence that the ACTA2+ fibrous cap originates from a tapestry of cell types, which transition to an MF-like state through distinct signalling pathways that are either dependent on or associated with extensive metabolic reprogramming.

xacerbated lesion development, suggesting that dysregulated EndoMT was detrimental for atherogenesis

SMC PDGFRβ signalling plays a critical role in SMC investment within the lesion and the fibrous cap. no significant differences in any of the indices of stability examined, including collagen content of the lesion and fibrous cap.

Indeed, sustained PDGFRβ signalling is essential for retention of ACTA2+ SMCs and collagen content within the fibrous cap, suggesting a critical protective role of PDGFRβ signalling in fibrous cap development and maintenance. Therefore, contrary to dogma, we propose that augmenting46 rather than reducing PDGFRβ signalling in SMCs during late-stage atherosclerosis would be a beneficial therapeutic strategy for maintaining lesion stability.

We have further shown that a PDGF/TGF-β-induced shift in bioenergetic pathway directly affects ECM synthesis in cultured SMCs that have phenotypically modulated to a MF-like state.



Indeed, sustained PDGFRβ signalling is essential for retention of ACTA2+ SMCs and collagen content within the fibrous cap, suggesting a critical protective role of PDGFRβ signalling in fibrous cap development and maintenance. Therefore, contrary to dogma, we propose that augmenting46 rather than reducing PDGFRβ signalling in SMCs during late-stage atherosclerosis would be a beneficial therapeutic strategy for maintaining lesion stability.

SMC PDGFRβ signalling plays a critical role in SMC investment within the lesion and the fibrous cap. no significant differences in any of the indices of stability examined, including collagen content of the lesion and fibrous cap.

Taken together, these data suggest that, in the absence of SMC investment, mesenchymal transitions of non-SMC-derived cells are capable of only temporarily maintaining indices of lesion stability.
\cite{newmanMultipleCellTypes2021}

PDGF::


    KLF4 is regulated by various signaling complexes at transcriptional and posttranslational levels.57,58 After exposure to PDGF-BB (platelet-derived growth factor BB), a stimulus of vSMC proliferation and phenotype switching, elevated levels of KLF4 were identified.

    Induction of KLF4 in vSMCs results in a phenotypic switch from contractile to mesenchymal-like and initiates the expression of mesenchymal markers such as stem cell antigen-1 (SCA1)/LY6A, CD34, and CD44.40,41 During this transition, while gaining expression of mesenchymal markers, the contractile vSMCs lose expression of their contractile markers.
    \cite{yapSixShadesVascular2021}


    -   Through the PDGFRb receptor, PDGFBB stimulates ELK1 phosphorylation, which competes with MYOCD for the same docking site on SRF, suppressing contractile gene expression. \cite{grootaertVascularSmoothMuscle2021}

    In addition to its increasingly well described role in tumorgenesis, PDGFR\beta signaling has also been implicated in the promotion of atherosclerosis \cite{andraeRolePlateletderivedGrowth2008, hePDGFRvSignallingRegulates2015}.

    cardiovascular disease: In general, two types of cells appear to respond in a pathological fashion to PDGFs—SMCs and fibroblasts—promoting vessel wall pathologies and fibrotic tissue scarring, respectively. Another general remark is that PDGFR- appears to be the dominant PDGFR involved in vascular pathology

    PDGF signaling in vascular disorders: PDGF signaling, especially signaling of PDGF-BB via PDGFR\beta has been implicated in the development of Artheriosklerose for a long time. But the molecular mechanism has escaped research until today. \cite{chenPlateletderivedGrowthFactors2013}
    \cite{andraeRolePlateletderivedGrowth2008} <- Review 2

    A number of adhesion molecules, cytokines and growth factors are implicated in the atherosclerotic lesion formation. These factors interact with each other and form an intricate network to affect tissue repair, cell proliferation and lipid metabolism. Among these factors that contribute to the development of atherosclerosis, an elevated PDGF expression has been detected in nearly all cell types of the atherosclerotic arterial wall and in the infiltrating inflammatory cells. All PDGFs, especially PDGF-A and PDGF-B, are detectable in atherosclerotic lesions. The PDGFR expression is also increased in the atherosclerotic vessel wall.
    \cite{huTargetingPlateletderivedGrowth2015}

    Shown in mouse model that PDGF signanling promotes artherosclerosis: PDGFRb pathway activation has a profound effect on vascular disease and support the conclusion that inflammation in the outer arterial layers is a driving process for atherosclerosis.

    By deletion of a STAT1-floxed allele in VSMCs, we show that PDGF-driven chemokine signalling and inflammation are STAT1 dependent. Finally, in the context of PDGF-driven atherosclerosis, deletion of STAT1 from VSMCs reduces plaque formation, demonstrating that inflammation of the arterial media and adventitia are important mechanisms by which PDGF signalling promotes atherosclerosis.
    \cite{hePDGFRvSignallingRegulates2015}


    Indeed, sustained PDGFRβ signalling is essential for retention of ACTA2+ SMCs and collagen content within the fibrous cap, suggesting a critical protective role of PDGFRβ signalling in fibrous cap development and maintenance. Therefore, contrary to dogma, we propose that augmenting46 rather than reducing PDGFRβ signalling in SMCs during late-stage atherosclerosis would be a beneficial therapeutic strategy for maintaining lesion stability.

    SMC PDGFRβ signalling plays a critical role in SMC investment within the lesion and the fibrous cap. no significant differences in any of the indices of stability examined, including collagen content of the lesion and fibrous cap.

    Taken together, these data suggest that, in the absence of SMC investment, mesenchymal transitions of non-SMC-derived cells are capable of only temporarily maintaining indices of lesion stability.
    \cite{newmanMultipleCellTypes2021}
