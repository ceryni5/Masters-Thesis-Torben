"Cardiovascular diseases (CVD) are the leading cause of death globally, taking an estimated 17.9 million lives each year.", according to the world health organization (WHO) \cite{}. The most common CVD is coronary artery disease (CAD) \cite{}, which is one of the significant challenges for future medical professionals. Investigating the molecular basics of this pathological process is pivotal for developing suitable treatment options.
While vascular smooth muscle cells (VSMC) are essential in initiating and maintaining atherosclerotic plaques \cite{}, the exact model for their exact role and function is frequently updated \cite{}. We hypothesize that extensive stimulation of VSMCs with pro-inflammatory cytokines such as PDGF-BB can cause oxidative stress and promote disease progression.
For this, we recapitulated that PDGF-BB boost of in vitro dedifferentiated HAoSMCs induces the generation of reactive oxygen species (ROS). Further, we tested the limits of the CellROX Assay to evaluate this interaction. We conclude that the assay provides a solid basis for the analysis of the generation of ROS due to PDGF-BB stimulation. Additionally, we suggest optimizations to the assay to improve its reproducibility.
In contrast to this hypothesis-driven approach, GWA studies are a tremendous observational tool for identifying exciting research targets \cite{}.
To complement CAD GWA study data, we curated and funneled genomic annotations of regulatory elements into a database. This database serves as the foundation of the GWAS Navigator. The GWAS Navigator is an interactive web-based visualization tool. It provides easy access to CAD GWA study summary statistics and their genomic context.
Finally, we integrated the curated data in an enrichment analysis. Testing for overlap of CAD risk variants with tissue- and cell-specific regulatory elements. We found overrepresentation overlaps mainly in biosamples from the heart, arteries, or lungs — a result that underlines the frequently described connection between heart- and lung diseases \cite{carterAssociationCardiovascularDisease2019, nowakLungFunctionCoronary2018, hanPulmonaryDiseasesHeart2007}.
