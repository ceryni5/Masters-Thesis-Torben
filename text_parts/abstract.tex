Placeholder, this is more or less what I am doing:

I am currently writing my Master thesis at the university of Lübeck at the \href{https://www.cardiogenetics-luebeck.de/}{\color{myblue}Institute for Cardiogenetics} on the topic of “Identification of genetic risk variants for atherosclerosis using oxidative stress assays in vascular smooth muscle cells and bioinformatic approaches”:

Coronary artery disease (CAD) describes the arterial build-up of fatty deposits to a point where the blood supply to the heart gets interrupted. It is one of the major causes of death worldwide. Risk factors for CAD are typical lifestyle factors like smoking or physical inactivity, but also include hereditary factors \cite{cdcCoronaryArteryDisease2021, CoronaryHeartDisease2018}. These can provide access to the molecular pathology of the disease. One amazing resource for studying these interactions are genome wide association studies (GWAS). Unfortunately, GWAS are just the first step in a longer journey of establishing causal loci to gene links, uncovering the molecular basis of disease, and implementing tools for clinical risk prediction. A plethora of follow-up analyses (postGWAS) can and need to be performed \cite{lichouFunctionalStudiesGWAS2020a}.

We hypothesize that oxidative stress in smooth muscle cells plays a role in stability of atherosclerotic plaques. For this reason, I am cultivating and differentiating primary human smooth muscle cells and characterizing them using oxidative stress assay, qPCR, seahorse assay \& immunofluorescence (IF).

Additionally, I am working with GWAS data on CAD \cite{aragamDiscoverySystematicCharacterization2021a}. Curating further publicly available data that can be used for bioinformatic follow-up analyses like the enrichment for involved tissues. Further, I am using the data to build a web application that allows co-visualization and visual exploration.
