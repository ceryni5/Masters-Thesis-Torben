“\Acp{cvd} are the leading cause of death globally, taking an estimated 17.9 million lives each year.”, according to the \ac{who} \cite{whoCardiovascularDiseases2022}. The most common CVD is \ac{cad} \cite{centersfordiseasecontrolandpreventionHeartDiseaseFacts2022}, which poses a significant challenge for current and future medical professionals. Investigating the molecular basics of this pathological process is pivotal for developing suitable treatment options.\\
While \acp{vsmc} are essential in initiating and maintaining atherosclerotic plaques \cite{doranRoleSmoothMuscle2008}, the exact model for their precise role and function is frequently updated \cite{liuSmoothMuscleCell2019, grootaertVascularSmoothMuscle2021, yapSixShadesVascular2021}. We hypothesize that extensive stimulation of \acp{vsmc} with pro-inflammatory cytokines such as ac{pdgf} can cause oxidative stress and promote disease progression.\\
For this, we recapitulated that \acs{pdgf} boost of in vitro dedifferentiated \ac{haosmc} induces the generation of \ac{ros}. Further, we tested the limits of the CellROX\texttrademark Assay to evaluate this interaction. We conclude that the assay provides a solid basis for analyzing \ac{ros} generation due to \ac{pdgf} stimulation. Additionally, we suggest optimizations to the assay to improve its reproducibility.\\
In contrast to this hypothesis-driven approach, \acp{gwas} are a tremendous observational tool for identifying exciting research targets \cite{uffelmannGenomewideAssociationStudies2021}. To complement \ac{cad} \ac{gwas} data, we curated and funneled genomic annotations of regulatory elements into a database. This database serves as the foundation of the \aca{gwas} Navigator, an interactive web-based visualization tool. It provides easy access to \ac{cad} \aca{gwas} GWA study summary statistics and their genomic context.\\
Finally, we integrated the curated data into an enrichment analysis — testing for overlap of \ac{cad} risk variants with tissue- and cell-specific regulatory elements. We found an overrepresentation of overlaps mainly in biosamples from the heart, arteries, or lungs — a result that underlines the frequently described connection between heart- and lung diseases \cite{carterAssociationCardiovascularDisease2019, nowakLungFunctionCoronary2018, hanPulmonaryDiseasesHeart2007}.
